%%%%%%%%%%%%%%%%%%%%%%%%%%%%%%%%%%%%%%%%%%%%%%%%%%%%%%%%%%%%%%%%%%%%%%
% Problem statement
\begin{statement}[
  problempoints=110,
  timelimit=2 sekunde,
  memorylimit=512 MiB,
]{Trobojnica}

  %\setlength\intextsep{-0.1cm}
%\begin{wrapfigure}[6]{r}{0.26\textwidth}
%\centering
%\includegraphics[width=0.26\textwidth]{img/flag.png}
%\end{wrapfigure}

,,\textit{...fool me once, shame on — shame on you. Fool me — you can't get fooled again."}''
-- W.

U ovom zadatku promatramo pravilne $N$-terokute kojima su stranice obojene u tri boje, a vrhovi označeni prirodnim brojevima u smjeru kazaljke na satu.
\textit{Triangulacija} je podjela mnogokuta na trokute unutarnjim
dijagonalama takva da dijagonale nemaju zajedničkih točaka osim vrhova mnogokuta te ne sijeku stranice mnogokuta osim u vrhovima mnogokuta.
Naravno, u ovom zadatku i svaka će dijagonala biti obojena u jednu od tri boje.

Triangulacija je \textit{domoljubna} ako za svaki od $N-2$ trokuta vrijedi da su
mu sve tri stranice različite boje. Vaš je zadatak odrediti čine li dijagonale triangulaciju, te je li ta triangulacija domoljubna.

%%%%%%%%%%%%%%%%%%%%%%%%%%%%%%%%%%%%%%%%%%%%%%%%%%%%%%%%%%%%%%%%%%%%%%
% Input
\subsection*{Ulazni podaci}
\textbf{U prvom je retku broj podzadatka za trenutni test primjer.} (Ukoliko vaše rješenje ne mari za podzadatke, samo učitajte  i ignorirajte ga.)
U drugom je retku prirodan broj $N$ iz teksta zadatka. \\
U trećem je retku $N$-teroznamenkasti broj čije znamenke predstavljaju boje
stranica $N$-terokuta u smjeru kazaljke na satu. Odnosno, prva znamenka
predstavlja boju stranice $(1,2)$, druga znamenka boju stranice $(2,3)$ i tako
sve do $N$-te znamenke koja predstavlja boju stranice $(N, 1)$. Dakako, boje su
označene znamenkama $1$, $2$ i $3$.
U svakom od sljedećih $N-3$ redaka nalazi se po jedna dijagonala u obliku $X$ $Y$ $C$, gdje su $X$ i $Y$ vrhovi dijagonale, a $C$ boja
$(1 \le X, Y \le N, 1 \le C \le 3)$. Svaki redak će opisivati validnu dijagonalu: vrhovi $X$ i $Y$ neće biti ni isti ni susjedni.

%%%%%%%%%%%%%%%%%%%%%%%%%%%%%%%%%%%%%%%%%%%%%%%%%%%%%%%%%%%%%%%%%%%%%%
% Output
\subsection*{Izlazni podaci}
Ako zadane dijagonale ne čine triangulaciju, ispišite "neispravna triangulacija".

Ako pak dijagonale čine triangulaciju, no ona nije domoljubna, ispišite "neispravno bojenje".

Ako dijagonale čine domoljubnu triangulaciju, ispišite "tocno".

%%%%%%%%%%%%%%%%%%%%%%%%%%%%%%%%%%%%%%%%%%%%%%%%%%%%%%%%%%%%%%%%%%%%%%
% Scoring
\subsection*{Bodovanje}
{\renewcommand{\arraystretch}{1.4}
  \setlength{\tabcolsep}{6pt}
  \begin{tabular}{ccl}
 Podzadatak & Broj bodova & Ograničenja \\ \midrule
  1 & 12 & $4 \le N \le 300$ \\
  2 & 17 & $4 \le N \le 2000$ \\
  3 & 23 & $4 \le N \le 2\cdot10^5$, odgovor je "neispravna triangulacija" ili "tocno" \\
  4 & 23 & $4 \le N \le 2\cdot10^5$, odgovor je "neispravno bojenje" ili "tocno" \\
  5 & 35 & $4 \le N \le 2\cdot10^5$
\end{tabular}}

Za razliku od zadatka "Trobojnica", ako vaš program točno ispisuje prvi redak u svakom testnom primjeru nekog podzadatka, osvojit će $10-\%$ bodova predviđenih za taj podzadatak.

%%%%%%%%%%%%%%%%%%%%%%%%%%%%%%%%%%%%%%%%%%%%%%%%%%%%%%%%%%%%%%%%%%%%%%
% Examples
\subsection*{Probni primjeri}
\begin{tabularx}{\textwidth}{X'X'X}
\sampleinputs{test/trobojnica.dummy.in.1}{test/trobojnica.dummy.out.1} &
\sampleinputs{test/trobojnica.dummy.in.2}{test/trobojnica.dummy.out.2} &
\sampleinputs{test/trobojnica.dummy.in.3}{test/trobojnica.dummy.out.3}
\end{tabularx}

%\textbf{Pojašnjenje drugog probnog primjera:}

%\textbf{1. upit} \textrightarrow{}
%$A_9 = 9$, $A_{10} = 1 + 0 = 1$, $A_{11} = 1 + 1 = 2$,
%$A_{12} = 1 + 2 = 3$, $A_{13} = 1 + 3 = 4$.\\
%\phantom{\textbf{1. upit} \textrightarrow{}}
%$A_9 + A_{10} + A_{11} + A_{12} + A_{13} = 9 + 1 + 2 + 3 + 4 = 19$.

%\textbf{2. upit} \textrightarrow{}
%$A_{44} = 4 + 4 = 8$, $A_{45} = 4 + 5 = 9$. $A_{44} + A_{45} = 8 + 9 = 17$.

%%%%%%%%%%%%%%%%%%%%%%%%%%%%%%%%%%%%%%%%%%%%%%%%%%%%%%%%%%%%%%%%%%%%%%
% We're done
\end{statement}

%%% Local Variables:
%%% mode: latex
%%% mode: flyspell
%%% ispell-local-dictionary: "croatian"
%%% TeX-master: "../hio.tex"
%%% End:
