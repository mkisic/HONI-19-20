%%%%%%%%%%%%%%%%%%%%%%%%%%%%%%%%%%%%%%%%%%%%%%%%%%%%%%%%%%%%%%%%%%%%%%
% Problem statement
\begin{statement}[
  problempoints=110,
  timelimit=1 second,
  memorylimit=512 MiB,
]{Trobojnica}

  %\setlength\intextsep{-0.1cm}
%\begin{wrapfigure}[6]{r}{0.26\textwidth}
%\centering
%\includegraphics[width=0.26\textwidth]{img/flag.png}
%\end{wrapfigure}

%\subsection

\textbf{Tvrdnja 1: } \; U svakoj triangulaciji za $N \ge 4$ postoje barem dva ``uha'', tj. trokut
koji dijeli dvije stranice s mnogokutom.

\emph{Skica dokaza: } \; Indukcija. Vrijedi za kvadrat.
Neka dijagonala dijeli mnogokut na dva manja mnogokuta. Ako je neki od njih trokut, to je traženo uho;
inače se možemo induktivno pozvati dalje. $\qed$

Prvo ćemo opisati postupak provjere triangulacije.
Naivni algoritam miče uha te uvijek održava trenutnu triangulaciju. To je moguće implementirati
u mnogim složenostima, pa je $O(N^2)$ implementacija bila dovoljna za prva dva podzadatka.
Ovdje dajemo jedan mogući $O(N \log n)$ algoritam.

Za svaku dijagonalu $ab$, napravimo usmjerene bridove $(a, b)$ i $(b, a)$. Sada svih $2(N-2)$
bridova sortirajmo po duljini iz početnog do krajnjeg vrha u smjeru kazaljke na satu.
Jasno je da prvi element sortiranog niza odgovara uhu.Može se dokazati da ako uha mićemo
tim redoslijedom, u svakom trenutku će prvi preostali brid u nizu odgovarati uhu
kojega trebamo maknuti u ovom koraku. (Naravno, ako dani brid ne čini uho, triangulacija
nije ispravna.)

Za detalje implementacije pogledajte službeno rješenje, gdje koristimo vezanu listu za održavanje
trenutnog poretka vanjskih vrhova mnogokuta. Tada su provjere je li neki brid čini uho i dobre
obojenosti uha jednostavne i mogu se raditi paralelno s danim algoritmom.

\end{statement}

%%% Local Variables:
%%% mode: latex
%%% mode: flyspell
%%% ispell-local-dictionary: "croatian"
%%% TeX-master: "../hio.tex"
%%% End:
