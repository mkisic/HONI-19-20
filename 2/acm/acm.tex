%%%%%%%%%%%%%%%%%%%%%%%%%%%%%%%%%%%%%%%%%%%%%%%%%%%%%%%%%%%%%%%%%%%%%%
% Problem statement
\begin{statement}[
  problempoints=50,
  timelimit=1 sekunda,
  memorylimit=512 MiB,
]{ACM}

\setlength\intextsep{-0.1cm}
\begin{wrapfigure}[10]{r}{0.16\textwidth}
\centering
\includegraphics[width=0.16\textwidth]{img/balloon.jpeg}
\end{wrapfigure}

Bliži se drevno (čitaj \textit{drveno}) programersko natjecanje za djecu i
mlade koje organizira nitko drugi doli ACM (\textit{Avijatičarski Centar Metković}).
Na natjecanju će nastupiti čak $N$ timova uzrasta do šest godina. Među timovima
je i zlatni trojac mladih avijatičara Hrvatske: Paula, Marin i Josip (obrnuto
abecedno, svaka sličnost sa stvarnim događajima i osobama nije slučajna).
Oblik natjecanja je standardan, dok kapetan posade radi piruete, kopilot čita
zadatke na ruskom jeziku i Morseovom abecedom diktira kod programeru koji se
nalazi izvan zrakoplova, ali je za njega sigurno pričvršćen ljepljivom vrpcom.

Timovi će na natjecanju ukrstiti koplja (preciznije krila) na $M$ različitih
zadataka. Timovi su na rang listi poredani silazno po broju riješenih
zadataka.

\begin{itemize}[topsep=0pt]
\item ,,\textit{Čekaj malo! Nisi objasnio ljudima kojim su redom poredani
timovi sa istim brojem riješenih zadataka!}'' -- dobacuje Marin kroz prozor
svojeg krilatog ljubimca.
\item ,,\textit{U pravu si, Marine.}'' -- odgovorih mu.
\end{itemize}

Timovi koji imaju isti broj riješenih zadataka poredani su \textbf{uzlazno} po
penalty vremenu. \textit{Penalty} vrijeme za neki tim je suma penalty vremena
svih točno riješenih zadataka. Penalty vrijeme točno riješenog zadatka
odgovara vremenu zadnjeg poslanog rješenja za taj zadatak kojem se pridodaje
\textbf{20 minuta} za svako pogrešno poslano rješenje na tom zadatku.  Tim
neće slati rješenje za zadatak koji su već točno riješili. Najveći dozvoljen
broj poslanih rješenja za isti zadatak je \textbf{9 po timu}. Ako timovi
imaju isti broj riješenih zadataka i isto penalty vrijeme, poredani su po
imenima abecedno.

Natjecanje traje \textbf{pet} sati. Tijekom prva četiri sata rang lista je
vidljiva svim timovima te za svaki tim pokazuje informacije o svakom zadatku
(koliko je ukupno slanja bilo, je li riješen i u koje vrijeme je riješen).
Tijekom ta četiri sata, poredak na listi se sa svakim slanjem automatski
ažurira.  Nakon četiri sata, lista se zamrzne, tj. ostane u poretku u kojem
je bila.  Informacije o točnosti rješenja poslanih tijekom zadnjeg sata svaki
tim zna samo za svoja vlastita, ali se za svaki tim i dalje za svaki zadatak
na listi ažurira koliko je ukupno rješenja poslano i kada je poslano zadnje.

Natjecanje je završilo, lista će se uskoro odmrznuti, a naš trojac, tj. tim s
imenom \texttt{NijeZivotJedanACM} treba vašu pomoć. Zanima ih koja je najniža
moguća pozicija na kojoj mogu završiti kada se lista odmrzne. Ali u tih pet
sati su se toliko izvrtjeli po modrom nebu da im je već zlo i programčić koji
ovo provjerava nisu u stanju napisati sami. Pomozite im!

%%%%%%%%%%%%%%%%%%%%%%%%%%%%%%%%%%%%%%%%%%%%%%%%%%%%%%%%%%%%%%%%%%%%%%
% Input
\subsection*{Ulazni podaci}
U prvom su retku prirodni brojevi $N$ $(1 \le N \le 1000)$ i $M$ $(1 \le M \le 15)$
iz teksta zadatka.

U sljedećih $N$ redaka je stanje zamrznute liste na kraju natjecanja. Svaki red
započinje imenom tima (riječ sastavljena od malih i velikih slova engleske
abecede ne duža od $20$ znakova, imena svih timova bit će različita) koje je
razmakom odvojeno od $M$ riječi koje su međusobno odvojene razmacima, a nose
informacije o rješenjima zadataka za taj tim, redom od prvog do zadnjeg
zadatka.

Riječi su za svaki zadatak oblika \texttt{SX/V}, gdje je:
\begin{itemize}[topsep=0pt]
  \item \texttt{S} stanje poslanih rješenja za taj zadatak (\texttt{‘+’}
    označava da je zadatak točno riješen, \texttt{‘-’} označava da nije, a
    \texttt{‘?’} označava da je zadnje rješenje poslano nakon zamrzavanja
    ljestvice).
  \item \texttt{X} je ukupan broj rješenja koja je taj tim poslao za taj zadatak
    te se izostavlja ako je jednak nuli.
  \item \texttt{V} je vrijeme u kojem je poslano zadnje rješenje. Vrijeme je u
    formatu \texttt{HH:MM:SS} (sa vodećim nulama) te je manje od $5$ sati.
    Cijeli \texttt{/V} dio se u riječi izostavlja ako zadatak nije točno
    riješen.
\end{itemize}

U posljednjem se retku nalazi odmrznuti redak za naš trojac, tim s imenom
\texttt{NijeZivotJedanACM}.

%%%%%%%%%%%%%%%%%%%%%%%%%%%%%%%%%%%%%%%%%%%%%%%%%%%%%%%%%%%%%%%%%%%%%%
% Output
\subsection*{Izlazni podaci}
U prvi i jedini redak ispišite najnižu moguću poziciju na kojoj naš trojac može
završiti nakon odmrzavanja liste.

%%%%%%%%%%%%%%%%%%%%%%%%%%%%%%%%%%%%%%%%%%%%%%%%%%%%%%%%%%%%%%%%%%%%%%
% Scoring
\subsection*{Bodovanje}
U test podacima ukupno vrijednima $20$ bodova, na ulazu se neće pojaviti znak
\texttt{‘?’}.

%%%%%%%%%%%%%%%%%%%%%%%%%%%%%%%%%%%%%%%%%%%%%%%%%%%%%%%%%%%%%%%%%%%%%%
% Examples
\subsection*{
  Probni primjeri
}
\begin{tabularx}{\textwidth}{X'X}
\sampleinputs{test/acm.dummy.in.1}{test/acm.dummy.out.1} &
\sampleinputs{test/acm.dummy.in.2}{test/acm.dummy.out.2}
\end{tabularx}

\begin{tabularx}{\textwidth}{X}
\sampleinputs{test/acm.dummy.in.3}{test/acm.dummy.out.3}
\end{tabularx}

\textbf{Pojašnjenje prvog probnog primjera:} \\
Lista će nakon odmrzavanja biti
ista kao i dok je bila zamrznuta, s našim trojcem na prvom mjestu!

\textbf{Pojašnjenje drugog probnog primjera:} \\
U najgorem će slučaju naš trojac
izgubiti samo od tima \texttt{StoJeZivot} i završiti na drugom mjestu.

\textbf{Pojašnjenje trećeg probnog primjera:} \\
U najgorem će slučaju naš trojac
izgubiti od timova \texttt{NisamSadaNistaDonio} i \texttt{JeLiMojKockaSeUmio} te
završiti na trećem mjestu.


%%%%%%%%%%%%%%%%%%%%%%%%%%%%%%%%%%%%%%%%%%%%%%%%%%%%%%%%%%%%%%%%%%%%%%
% We're done
\end{statement}

%%% Local Variables:
%%% mode: latex
%%% mode: flyspell
%%% ispell-local-dictionary: "croatian"
%%% TeX-master: "../hio.tex"
%%% End:
