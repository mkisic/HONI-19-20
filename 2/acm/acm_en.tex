%%%%%%%%%%%%%%%%%%%%%%%%%%%%%%%%%%%%%%%%%%%%%%%%%%%%%%%%%%%%%%%%%%%%%%
% Problem statement
\begin{statement}[
  problempoints=50,
  timelimit=1 second,
  memorylimit=512 MiB,
]{ACM}

\setlength\intextsep{-0.1cm}
\begin{wrapfigure}[10]{r}{0.16\textwidth}
\centering
\includegraphics[width=0.16\textwidth]{img/balloon.jpeg}
\end{wrapfigure}


An ancient programming competition is getting near and it is organized by
none other than ACM (\textit{Aeronautic Centre of Metković}). Exactly
$N$ teams will compete for the grand prize and among them is a golden
Croatian trio: Paula, Marin and Josip. The contest format is standard,
while the pilot is performing aerobatic maneuvers, the co-pilot reads the
problem statements and attempts to transmit the solutions to the
\sout{code monkey} main programmer who is securely taped to the aircraft's
exterior.

The contest consists of $M$ different tasks and the teams are (non-increasingly)
ordered by the number of solved tasks.

%\begin{itemize}[topsep=0pt]
%\item ``\textit{Wait a minute! You didn't explain what happens to the standings
  %when certain teams solve the same number of tasks!}'' -- yells Marin through
  %the airplane window.
%\item ``\textit{You're right, Marin.}'' -- I replied.
%\end{itemize}

The teams that have the same number of solved tasks are ordered
(non-decreasingly) by so-called penalty time. \textit{Penalty time} of a
certain team is the sum of penalty time they achieved on each of
their correctly solved tasks.  Penalty time of a correctly solved task equals
to the time it took for the team to solve that task (from the beginning of
the contest) increased by additional \textbf{20 minutes} for each of the
wrongly submitted solutions for that task. No teams will attempt to submit a
solution for a problem that they have already solved and the maximum number
of submissions for a certain task is \textbf{9 for each team}.  If some teams
have the same number of solved problems and the same penalty time, they will
be ordered alphabetically in the final standings.

The competition lasts for \textbf{five hours}. During the first four hours the
standings are available to all teams and contain information about the status of
each task for each team (number of submissions, whether it was solved and when).
During those four hours, the order of the teams will be automatically updated
after each submission. In the last hour though, the standings become frozen, i.e.,
the order of the teams is not updated after a new submission is judged. During
that time, each team knows the judgement of their own submissions, but they
don't know the judgement of submissions made by other teams. They only know which
tasks were submitted by other teams, how many times were they submitted and when
was the last submission for each task.

The contest is over and the standings should be unfrozen soon. Our heroes, team
named \texttt{NijeZivotJedanACM} need your help. They want to know what is the
worst possible position on the scoreboard they could end up in after the standings
become unfrozen. Help them!

%%%%%%%%%%%%%%%%%%%%%%%%%%%%%%%%%%%%%%%%%%%%%%%%%%%%%%%%%%%%%%%%%%%%%%
% Input
\subsection*{Input}
The first line contains integers $N$ $(1 \le N \le 1000)$ and $M$ $(1 \le M \le 15)$
from the task description.

The next $N$ lines represent the frozen standings. Each line begins by a team
name (string made up of lowercase and uppercase English letters which consists
of at most $20$ characters, names of all teams will be different) which is
separated by a space from $M$ (also space-separated) strings that hold
the information about the status of each task for that team.

Those strings are of the form \texttt{SX/V}, where:
\begin{itemize}[topsep=0pt]
  \item \texttt{S} represents the status of the task -- \texttt{‘+’} means the
    task is solved correctly, \texttt{‘-’} means it is solved incorrectly and
    \texttt{‘?’} means that the last submission was sent when the standings were
    already frozen.
  \item \texttt{X} represents the number of submissions that were sent by that
        team for this specific task. It is omitted if no submissions were made
        for that task.
  \item \texttt{V} represents the time when the last submission was sent by
        that team for this specific task. It is given in the format
        \texttt{HH:MM:SS} (with leading zeroes) and is less than $5$ hours.
    The whole \texttt{/V} part is omitted if the task is not solved correctly
    (status \texttt{'-'})
\end{itemize}

The last line contains the unfrozen standings of our heroes, team named
\texttt{NijeZivotJedanACM}.

%%%%%%%%%%%%%%%%%%%%%%%%%%%%%%%%%%%%%%%%%%%%%%%%%%%%%%%%%%%%%%%%%%%%%%
% Output
\subsection*{Output}
In the first and only line you should output the worst possible position in
the standings where our heroes might end up in after the standings become
unfrozen.

%%%%%%%%%%%%%%%%%%%%%%%%%%%%%%%%%%%%%%%%%%%%%%%%%%%%%%%%%%%%%%%%%%%%%%
% Scoring
\subsection*{Scoring}
In test cases worth a total of $20$ points, the input won't contain the
\texttt{‘?’} character.

%%%%%%%%%%%%%%%%%%%%%%%%%%%%%%%%%%%%%%%%%%%%%%%%%%%%%%%%%%%%%%%%%%%%%%
% Examples
\subsection*{Examples}
\begin{tabularx}{\textwidth}{X'X}
\sampleinputs{test/acm.dummy.in.1}{test/acm.dummy.out.1} &
\sampleinputs{test/acm.dummy.in.2}{test/acm.dummy.out.2}
\end{tabularx}

\begin{tabularx}{\textwidth}{X}
\sampleinputs{test/acm.dummy.in.3}{test/acm.dummy.out.3}
\end{tabularx}

\textbf{Clarification of the fist example:} \\
Nothing will change after the standings our unfrozen. Therefore,
our heroes will remain in the first place!

\textbf{Clarification of the second example} \\
In the worst case our heroes will only lose from the team \texttt{StoJeZivot},
thereby finishing in second place.

\textbf{Clarification of the third example} \\
In the worst case our heroes will lose from teams
U najgorem će slučaju naš trojac
\texttt{NisamSadaNistaDonio} and \texttt{JeLiMojKockaSeUmio}, thereby
finishing in third place.


%%%%%%%%%%%%%%%%%%%%%%%%%%%%%%%%%%%%%%%%%%%%%%%%%%%%%%%%%%%%%%%%%%%%%%
% We're done
\end{statement}

%%% Local Variables:
%%% mode: latex
%%% mode: flyspell
%%% ispell-local-dictionary: "croatian"
%%% TeX-master: "../hio.tex"
%%% End:
