%%%%%%%%%%%%%%%%%%%%%%%%%%%%%%%%%%%%%%%%%%%%%%%%%%%%%%%%%%%%%%%%%%%%%%
% Problem statement
\begin{statement}[
  problempoints=30,
  timelimit=1 sekunda,
  memorylimit=512 MiB,
]{Radio}

\setlength\intextsep{-0.1cm}
\begin{wrapfigure}[7]{r}{0.23\textwidth}
\centering
\includegraphics[width=0.23\textwidth]{img/radio.jpg}
\end{wrapfigure}

Stjepan je, nakon godinu dana rada u poznatoj hrvatskoj firmi, kupio polovni
BMW. Na putu od Zagreba do Belice (malog mjesta pored Pribislavca) Stjepan je
glasno puštao glazbu i nakon \textbf{dugo dugo} trpljenja te buke njegova
cura Ana mu je uputila $N$ pritužbi.

Svaka pritužba je bila oblika:
,,Sjepane, molim te stišaj glazbu za $A_i$ jedinica''.  Stjepan bi rado
poslušao svoju curu, ali također bi htio održati visoku razinu glasnoće
glazbe. Kako bi pomirio svoje dvojbe Stjepan je odlučio povećati jačinu
glazbe za $B_i$ jedinica svaki put kad bi mu Ana uputila pritužbu, a zatim
ispuniti želju svojoj djevojci i smanjiti jačinu glazbe za $A_i$ jedinica.

Ako znamo da je jačina glazbe na početku puta bila $X$ jedinica, pitamo se
kolika je bila na kraju puta?

%%%%%%%%%%%%%%%%%%%%%%%%%%%%%%%%%%%%%%%%%%%%%%%%%%%%%%%%%%%%%%%%%%%%%%
% Input
\subsection*{Ulazni podaci}
U prvom su retku dva prirodna broja $N$ i $X$ $(1 \le N, X \le 10^5)$ iz teksta
zadatka. \\
U sljedećih su $N$ redaka dva prirodna broja $A_i$ i $B_i$ $(1 \le A_i, B_i \le 10^5)$
iz teksta zadatka.

Jačina glazbe tokom puta nikada neće biti negativna.

%%%%%%%%%%%%%%%%%%%%%%%%%%%%%%%%%%%%%%%%%%%%%%%%%%%%%%%%%%%%%%%%%%%%%%
% Output
\subsection*{Izlazni podaci}
U jedini redak ispišite jačinu glazbe na kraju puta.

%%%%%%%%%%%%%%%%%%%%%%%%%%%%%%%%%%%%%%%%%%%%%%%%%%%%%%%%%%%%%%%%%%%%%%
% Scoring
\subsection*{Bodovanje}
U test podacima ukupno vrijednima $15$ bodova vrijedit će da je $N = 1$.

%%%%%%%%%%%%%%%%%%%%%%%%%%%%%%%%%%%%%%%%%%%%%%%%%%%%%%%%%%%%%%%%%%%%%%
% Examples
\subsection*{Probni primjeri}
\begin{tabularx}{\textwidth}{X'X'X}
\sampleinputs{test/radio.dummy.in.1}{test/radio.dummy.out.1} &
\sampleinputs{test/radio.dummy.in.2}{test/radio.dummy.out.2} &
\sampleinputs{test/radio.dummy.in.3}{test/radio.dummy.out.3}
\end{tabularx}

\textbf{Pojašnjenje prvog probnog primjera:}
Na početku vožnje jačina glazbe je $10$ jedinica. Nakon što Ana uputi prvu i
jedinu pritužbu, Stjepan pojača radio za pet jedinica na $15$, a zatim ga
smanji za jednu jedinicu na $14$.

\textbf{Pojašnjenje drugog probnog primjera:}
Na početku vožnje jačina glazbe je $7$ jedinica, nakon prve pritužbe jačina je
$4$ jedinice, a nakon druge i posljednje pritužbe jačina glazbe iznosi $2$
jedinice.

%%%%%%%%%%%%%%%%%%%%%%%%%%%%%%%%%%%%%%%%%%%%%%%%%%%%%%%%%%%%%%%%%%%%%%
% We're done
\end{statement}

%%% Local Variables:
%%% mode: latex
%%% mode: flyspell
%%% ispell-local-dictionary: "croatian"
%%% TeX-master: "../hio.tex"
%%% End:
