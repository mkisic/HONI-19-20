%%%%%%%%%%%%%%%%%%%%%%%%%%%%%%%%%%%%%%%%%%%%%%%%%%%%%%%%%%%%%%%%%%%%%%
% Problem statement
\begin{statement}[
  problempoints=110,
  timelimit=1 sekunda,
  memorylimit=512 MiB,
]{Popcount}

\setlength\intextsep{-0.1cm}
\begin{wrapfigure}[3]{r}{0.22\textwidth}
\centering
\includegraphics[width=0.22\textwidth]{img/computer.png}
\end{wrapfigure}

\textit{\textbf{M}alo \textbf{Al}gebarsko \textbf{Na}predno \textbf{R}ačunalo}
(popularno \texttt{MALNAR}) posljednji je krik tehnologije u svijetu
minijaturnih programabilnih računala. Vlastite programe za ovaj uređaj moguće
je pisati u programskom jeziku \textit{MalnarScript} kojeg krase sljedeća
obilježja:

\begin{itemize}[topsep=0pt]
   \item Ulaz u program je jedan nenegativan cijeli broj koji je strogo manji
         od $2^N$.
   \item Izlaz programa je jedan nenegativan cijeli broj koji je strogo manji
         od $2^N$.
   \item U izvornom kodu \textit{MalnarScript}-a moguće je baratati točno
         jednom $N$-bitnom cjelobrojnom  varijablom $A$ bez predznaka (engl.\
         \textit{unsigned}). Na početku programa vrijednost varijable $A$
         odgovara ulazu u program, a njena završna vrijednost predstavlja izlaz
         programa.
   \item Izvorni kod sastoji se od najviše $K$ naredbi oblika \texttt{A=<izraz>}
         koje se slijedno izvršavaju te se svaka od njih mora sastojati od
         \textbf{najviše 1000 znakova}. Simbol \texttt{<izraz>} definiramo
         rekurzivno kao:
\end{itemize}

\begin{center}
  \texttt{<izraz> = A | <broj> | (<izraz><operator><izraz>)}
\end{center}

Odnosno, simbol \texttt{<izraz>} može biti varijabla $A$, može odgovarati
simbolu \texttt{<broj>} ili pak može unutar okruglih zagrada predstavljati
dvočlani izraz u kojem se svaki operand pokorava istoj definiciji.

Simbol \texttt{<broj>} u gornjoj definiciji predstavlja nenegativan cijeli broj
u dekadskom zapisu koji je strogo manji od $2^N$, dok simbol
\verb|<operator>| može biti \verb|+|, \verb|-|, \verb|||,
\verb|&|, \verb|<<| ili \verb|>>| te redom može predstavljati
operacije zbrajanja, oduzimanja, bitovnog ili, bitovnog i, posmaka ulijevo i
posmaka udesno.  Također, znak $A$ se u izrazu smije pojavljivati
\textbf{najviše 5 puta}.

U slučaju da prilikom zbrajanja ili oduzimanja dođe do aritmetičkog
prelijevanja (engl.\ \textit{overflow}) ili podlijevanja (engl.\
\textit{underflow}), programski jezik \textit{MalnarScript} će te operacije
izvesti modulo $2^N$. Primjerice, uz pretpostavku $N=3$ izraz $(7+3)$ u jeziku
\textit{MalnarScript} iznosi $2$, a izraz $(2-5)$ iznosi $5$.

Desna strana jednakosti svake naredbe u konačnici se evaluira u jedan broj koji
se tada sprema u varijablu $A$. Evaluacija samog izraza provodi se tako da se
najprije svako pojavljivanje znaka $A$ zamijeni sa trenutnom vrijednošću
varijable $A$. Zatim se izraz računa kao u matematici, odnosno, najprije se
rješavaju zagrade. Primijetite da je definicija ispravnog izraza takva da je
prioritet operatora nebitan, odnosno, račun je u potpunosti definiran
zagradama.

Vaš je zadatak napisati program koji ispisuje program u jeziku
\textit{MalnarScript} koji računa broj jedinica u binarnom zapisu ulaza u
program.

%%%%%%%%%%%%%%%%%%%%%%%%%%%%%%%%%%%%%%%%%%%%%%%%%%%%%%%%%%%%%%%%%%%%%%
% Input
\subsection*{Ulazni podaci}
U prvom su retku prirodni brojevi $N$ i $K$ $(1 \le N, K \le 500)$ iz teksta
zadatka.

%%%%%%%%%%%%%%%%%%%%%%%%%%%%%%%%%%%%%%%%%%%%%%%%%%%%%%%%%%%%%%%%%%%%%%
% Output
\subsection*{Izlazni podaci}
U prvom retku potrebno je ispisati broj naredbi traženog programa u jeziku
\textit{MalnarScript}.

U preostalim je retcima potrebno ispisati naredbe traženog programa. Sve
naredbe moraju u potpunosti poštovati sintaksu jezika \textit{MalnarScript}
kako je opisano u tekstu zadatka.

%%%%%%%%%%%%%%%%%%%%%%%%%%%%%%%%%%%%%%%%%%%%%%%%%%%%%%%%%%%%%%%%%%%%%%
% Scoring
\subsection*{Bodovanje}
{\renewcommand{\arraystretch}{1.4}
  \setlength{\tabcolsep}{6pt}
  \begin{tabular}{ccl}
 Podzadatak & Broj bodova & Ograničenja \\ \midrule
  1 & 15 & $1 \le N \le 100$, $K = N$ \\
  2 & 15 & $N = 500$, $K = 100$ \\
  3 & 35 & $1 \le N \le 32$, $K=5$ \\
  4 & 45 & $100 \le N \le 500$, $K=10$
\end{tabular}}

%%%%%%%%%%%%%%%%%%%%%%%%%%%%%%%%%%%%%%%%%%%%%%%%%%%%%%%%%%%%%%%%%%%%%%
% Examples
\subsection*{Probni primjeri}
\begin{tabularx}{\textwidth}{X'X}
\sampleinputs{test/popcount.dummy.in.1}{test/popcount.dummy.out.1} &
\sampleinputs{test/popcount.dummy.in.2}{test/popcount.dummy.out.2}
\end{tabularx}

\textbf{Pojašnjenje prvog probnog primjera:} \\
$ulaz=0 \Rightarrow izlaz=(0-((0\&2)>>1))=(0-(0>>1))=(0-0)=0$\\
$ulaz=1 \Rightarrow izlaz=(1-((1\&2)>>1))=(1-(0>>1))=(1-0)=1$\\
$ulaz=2 \Rightarrow izlaz=(2-((2\&2)>>1))=(2-(2>>1))=(2-1)=1$\\
$ulaz=3 \Rightarrow izlaz=(3-((3\&2)>>1))=(3-(2>>1))=(3-1)=2$

%%%%%%%%%%%%%%%%%%%%%%%%%%%%%%%%%%%%%%%%%%%%%%%%%%%%%%%%%%%%%%%%%%%%%%
% We're done
\end{statement}

%%% Local Variables:
%%% mode: latex
%%% mode: flyspell
%%% ispell-local-dictionary: "croatian"
%%% TeX-master: "../hio.tex"
%%% End:
