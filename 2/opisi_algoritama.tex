\documentclass[a4paper]{article}
\usepackage{zadaci}
\usepackage{wrapfig}
\usepackage{url}
\usepackage{tikz}
\usepackage{amsmath}
\usepackage[normalem]{ulem}
\usetikzlibrary{angles,quotes}
\contestname{Hrvatsko otvoreno natjecanje u informatici\\1.\ kolo, 19. listopada 2019.}
\markright{\textbf{\textsf{Opisi algoritama}}}

\begin{document}

\section*{Opisi algoritama}
Zadatke, testne primjere i rješenja pripremili: Nikola Dmitrović, Karlo Franić,
Gabrijel Jambrošić, Marin Kišić, Josip Klepec, Vedran Kurdija, Daniel Paleka,
Ivan Paljak, Stjepan Požgaj i Paula Vidas. Primjeri implementiranih rješenja su
dani u priloženim izvornim kodovima.

\subsection*{Zadatak: Osijek}
\textsf{Pripremio: Nikola Dmitrović}\\
\textsf{Potrebno znanje: naredba učitavanja i ispisivanja, operacija dijeljenja}

Rješenje ovisi o ostatku pri dijeljenju broja $N$ s $K$. Ako su oni djeljivi bez
ostatka, očito je da će Lega točno ,,$N$ podijeljeno s $K$'' puta izvući $K$ komada
čipsa iz kutije s $N$ komada. Ako nisu djeljivi bez ostatka tada će u kutiji,
nakon što Lega iz nje ,,$N$ podijeljeno s $K$'' puta izvuče $K$ komada, ostati još
čipsa koji će se moći izvaditi u još jednom vađenju.

\textit{Programski kod (pisan u \texttt{Python 3}):}

\vspace{-2ex}
\begin{verbatim}
N = int(input())
K = int(input())

if N % K == 0:
  print(N // K)
else:
  print(N // K + 1)
\end{verbatim}

ili

\begin{verbatim}
N = int(input())
K = int(input())
print(N // K + (N % K != 0))
\end{verbatim}

\subsection*{Zadatak: Radio}
\textsf{Pripremili: Karlo Franić i Marin Kišić}\\
\textsf{Potrebno znanje: naredba ponavljanja, osnovne aritmetičke operacije,
poznavanje tipova podataka za spremanje velikih brojeva}

Promotrimo najprije parcijalu koja nosi $15$ bodova. Unosimo tri broja $X$, $A$
i $B$.  Stjepan će u tom slušaju pojačati radio za $B$ jedinica na $X + B$, a
potom ga smanjiti za $A$ jedinica na $X + B - A$. Dakle, za osvajanje prve
parcijale bilo je dovoljno ispisati $X + B - A$.

Za osvajanje preostalih bodova potrebno je primijetiti da će se za svaki Anin
upit početna jačina promijeniti za $B_i - A_i$ jedinica. Petljom prolazimo kroz
sve Anine upite i trenutnoj jačini dodajemo $Bi - Ai$ za $i$-ti upit.

Promatrajmo slučaj u kojemu je $N$ maksimalan, svaki $B$ maksimalan, a svaki
$A$ minimalan. Tada će konačna jačina glazbe biti $N \cdot MAXB + X = 100000 *
100000 + X$. Primjetimo da taj broj više ne stane u klasičan $32$-bitni
cjelobrojni tip podatka (npr. \texttt{int} u jezicima \texttt{C/C++}). Da
biste uspjeli spremiti rješenje potrebno je koristiti barem $64$-bitni
cjelobrojni tip podatka (npr. \texttt{long long} u jezicima \texttt{C/C++}).

Između ostalog, ovim su zadatkom autori odlučili provjeriti koliko je
natjecatelja, obzirom na rezultate drugog zadatka prethodnog kola, naučilo
ispravno koristiti brojčane tipove podataka u svojim omiljenim programskim
jezicima.

\subsection*{Zadatak: ACM}
\textsf{Pripremili: Vedran Kurdija i Marin Kišić}\\
\textsf{Potrebno znanje: naredba učitavanja, naredba ponavljanja, rad sa
stringovima, sortiranje}

Za $20$ bodova bilo je dovoljno zaključiti da će, budući da na ulazu nema
\texttt{‘?’}, lista nakon odmrzavanja izgledati isto kao i zamrznuta lista.
Dakle, trebalo je pronaći u kojem se retku zamrznute liste nalazi tim
NijeZivotJedanACM i ispisati indeks tog retka.

Za preostalih $30$ bodova, trebalo je zaključiti da su, u najgorem slučaju za
naš tim, svi ostali timovi točno riješili sve zadatke sa \texttt{‘?’}, tj.
zadatke poslane dok je lista bila zamrznuta. Intuicija je jasna, što više
točnih rješenja imaju drugi timovi, to gore za naš tim. Zato, u zamrznutoj
listi možemo za sve timove osim našega pretvoriti sve \texttt{‘?’} u
\texttt{‘+’}. Ovime smo dobili odmrznutu listu u najgorem slučaju za naš tim.
Sada nas zanima koliko će se timova u tako odmrznutoj listi naći iznad našeg
tima.

Najprije ćemo iz odmrznutog retka našeg tima koji smo dobili na ulazu
prebrojiti broj točno riješenih zadataka te izračunati \textit{penalty}
vrijednost našeg tima.  Uvest ćemo pomoćnu varijablu pozicija sa početnom
vrijednošću $1$, na koju ćemo dodati koliko je boljih timova od našeg.

Petljom ćemo obići sve timove u odmrznutoj listi (osim našeg) te za svaki
prebrojiti koliko točno riješenih zadataka imaju, kao i kolika je njihova
\textit{penalty} vrijednost. Ako neki tim ima više riješenih zadataka od
našega, povećat ćemo varijablu \texttt{pozicija} za $1$. Ako neki tim ima
jednak broj riješenih zadataka kao naš, usporedit ćemo \textit{penalty}
vrijednosti tog i našeg tima te ako taj tim ima manju \textit{penalty}
vrijednost, uvećati varijablu \texttt{pozicija} za $1$. Ako neki tim pak ima i
jednak broj riješenih zadataka kao naš i jednaku \textit{penalty} vrijednost
kao naš, usporedit ćemo imena timova. Ako je ime tog tima prije po abecedi,
uvećat ćemo varijablu \texttt{pozicija} za $1$. Na kraju ćemo ispisati
varijablu \texttt{pozicija}.

\subsection*{Zadatak: Slagalica}
\textsf{Pripremio: Gabrijel Jambrošić}\\
\textsf{Potrebno znanje: analiza slučajeva, pohlepni algoritmi}

Prvu parcijalu moguće je riješiti jednostavnom analizom slučajeva koristeći par
\texttt{if} naredbi.

Drugu parcijalu moguće je riješiti isprobavanjem svih permutacija u vremenskoj
složenosti $\mathcal{O}(n!)$. Implementaciju ovog rješenja možete vidjeti u
izvornom kodu \texttt{slagalica\_nfact.cpp}

Primijetimo da ćemo u trećoj parcijali naizmjence stavljati komadiće oblika $1$
i $4$ pa je optimalno uvijek uzeti onaj s najmanjim brojem kojeg možemo staviti.
Ako nam na kraju ostane višak, rješenje ne postoji.

Za četvrtu parcijalu pretpostavimo da smo složili neke komadiće te zadnji kojeg
smo stavili na desnom kraju ima udubinu. Tada sljedeći komadić kojeg ćemo
staviti na lijevom kraju mora imati izbočinu, a taj uvjet zadovoljavaju
komadići oblika $1$ i $2$. Ako stavimo komadić oblika $2$, na desnoj će strani
opet biti udubina te ćemo opet trebati komadić oblika $1$ ili $2$. S druge
strane, ako stavimo komadić oblika $1$, onda ćemo u sljedećem koraku trebati
komadiće oblika $3$ ili $4$. Drugim riječima, komadić oblika $1$ mijenja
traženi oblik sljedećeg komadića, a isto vrijedi i za komadić oblika $4$. Stoga
ćemo u ovom podzadatku najprije postaviti sve komadiće oblika $2$ (ili $3$,
ovisno o obliku početnog komadića), zatim ćemo staviti komadić oblika $1$ ili
$4$, a onda ćemo postaviti sve oblika $3$ (ili $2$). Uvijek postavljamo komadić
traženog oblika s najmanjim brojem jer će tada i niz biti najmanji. Ako na
kraju ne možemo spojiti zadnji komadić, rješenje ne postoji. Dodatno, moramo
paziti na slučaj kada ne postoje komadići oblika $1$ i $4$.

Za potpuno rješenje zadatka, uzmimo na početku jedan naivan pohlepni algoritam
kojim uvijek uzimamo komadić traženog oblika s najmanjim brojem. Problem je taj
da na kraju mogu ostati komadići oblika $2$ ili $3$ (ako
ostanu komadići oblika $1$ ili $4$ rješenje ne postoji) koje ne možemo više
staviti, a prije smo ih mogli. To možemo riješiti na način da te komadiće
stavimo na zadnje mjesto na koje smo ih još uvijek mogli staviti jer će tamo
najmanje povećati konačan niz. Možemo primijetiti da će to mjesto biti
neposredno ispred zadnjeg komadića oblika $1$ ili $4$ jer smo tamo posljednji
put promijenili traženi oblik komadića.

\subsection*{Zadatak: Checker}
\textsf{Pripremili: Paula Vidas i Daniel Paleka}\\
\textsf{Potrebno znanje: matematička indukcija, vezana lista}

\textbf{Tvrdnja 1: } \; U svakoj triangulaciji za $N \ge 4$ postoje barem dva
``uha'', tj. trokut koji dijeli dvije stranice s mnogokutom.

\emph{Skica dokaza: } \; Indukcija. Vrijedi za kvadrat.  Neka dijagonala dijeli
mnogokut na dva manja mnogokuta. Ako je neki od njih trokut, to je traženo uho;
inače se možemo induktivno pozvati dalje.

Prvo ćemo opisati postupak provjere triangulacije.  Naivni algoritam miče uha
te uvijek održava trenutnu triangulaciju. To je moguće implementirati u mnogim
složenostima, pa je $\mathcal{O}(n^2)$ implementacija bila dovoljna za prva dva
podzadatka.  Ovdje dajemo jedan mogući $\mathcal{O}(n \log n)$ algoritam.

Za svaku dijagonalu $ab$, napravimo usmjerene bridove $(a, b)$ i $(b, a)$. Sada
svih $2(N-2)$ bridova sortirajmo po duljini iz početnog do krajnjeg vrha u
smjeru kazaljke na satu.  Jasno je da prvi element sortiranog niza odgovara
uhu. Može se dokazati da ako uha mićemo tim redoslijedom, u svakom trenutku će
prvi preostali brid u nizu odgovarati uhu kojega trebamo maknuti u ovom koraku.
(Naravno, ako dani brid ne čini uho, triangulacija nije ispravna.)

Za detalje implementacije pogledajte službeno rješenje, gdje koristimo vezanu
listu za održavanje trenutnog poretka vanjskih vrhova mnogokuta. Tada su
provjere je li neki brid čini uho i dobre obojenosti uha jednostavne i mogu se
raditi paralelno s danim algoritmom.

\subsection*{Zadatak: Popcount}
\textsf{Pripremio: Ivan Paljak}\\
\textsf{Potrebno znanje: konstruktivni algoritmi, turnirsko stablo}

Nepoznat programski jezik, dopuštena samo jedna varijabla, ograničenja na broj
naredbi i duljinu naredbe, \dots -- savršena kombinacija za rješavanje zadatka
"podzadatak po podzadatak".

U prvom podzadatku dopušteno nam je koristiti broj naredbi koji je (skoro)
jednak broju bitova u ulaznoj vrijednosti. Ovo nam sugerira da ćemo zadatak
rješavati "bit po bit". Pretpostavimo da smo sufiks od $X$ bitova uspješno
riješili, odnosno, taj sufiks varijable $A$ sadrži vrijednost koja odgovara
broju jedinica u tom istom sufiksu ulazne vrijednosti dok ostali
bitovi nisu promijenjeni u odnosu na bitove ulazne vrijednosti. Primijetimo da
sama ulazna vrijednost u program već ima riješen sufiks veličine $1$. Sufiks od
$X+1$ bitova ćemo riješiti tako da vrijednost $(X+1)$-og bita pribrojimo
varijabli $A$ te taj bit postavimo na $0$. To možemo napraviti naredbom
\verb|A=((A&(((1<<N)-1)-(1<<X)))+((A&(1<<X))>>X))|. Koristeći naredbe ovog
oblika za svaki bit uspješno smo riješili prvi podzadatak.

Za osvajanje bodova na drugom podzadatku dovoljno je iskoristiti ideju iz prvog
podzadatka zajedno sa svojstvom jezika da se varijabla $A$ smije pojavljivati
najviše $5$ puta. Naravno, umjesto "bit po bit", zadatak ćemo rješavati
tehnikom "četiri bita po četiri bita" i tako smanjiti broj naredbi $4$ puta.

Za rješavanje preostalih podzadataka bilo je potrebno sjetiti se jedne poznate
strukture podataka -- \textit{turnirskog stabla}. Zamislimo da je nad našim
bitovima izgrađeno turnirsko stablo u kojem svaki čvor pamti sumu bitova u svom
podstablu. Umjesto u dekadskom brojevnom sustavu, zamislimo da je u svakom čvoru
turnirskog stabla napisana vrijednost u binarnom sustavu te da je prikazana u
onoliko bitova kolika je veličina intervala kojeg taj čvor pokriva.

Slijepimo li vrijednosti svih listova (slijeva nadesno), dobit ćemo ulaz u
program.  Slijepimo li vrijednosti njihovih roditelja (slijeva nadesno), dobit
ćemo željenu vrijednost varijable $A$ nakon prve naredbe. Ponavljamo li ovaj
postupak sve do korijena stabla u končnici ćemo u varijabli $A$ imati
točno rješenje, a broj naredbi bit će reda $\mathcal{O}(\log n)$. Ostaje nam
samo osmisliti prijelaz između redaka u turnirskom stablu.

U prvoj naredbi trebamo uzeti sve bitove na parnim pozicijama te im pribrojiti
posmaknute bitove na neparnim pozicijama. To možemo dobiti izrazom
\verb|((...01010101&A)+((...10101010&A)>>1))|. U sljedećem koraku algoritma
potrebno je riješiti intervale duljine četiri pa je odgovarajuči izraz
\verb|((...00110011&A)+((...11001100&A)>>2))|. Ponavljanjem ovog postupka
dobivamo konačno rješenje zadatka.

Ako ste konstantne vrijednosi iz gornjih izraza eksplicitno izračunali, osvojili
ste svih $110$ bodova na zadatku. U slučaju da ste to napraviili implicitno,
vrlo ste lagano mogli prekoračiti broj dopuštenih znakova u jednoj naredbi pa
biste osvojili bodove samo na trećoj parcijali.

\subsection*{Zadatak: Zvijezda}
\textsf{Pripremio: Marin Kišić}\\
\textsf{Potrebno znanje: ccw funkcija, binarna pretraga, analiza problema}

Neka je s $A_i$ označena $i$-ta točka $N$-terokuta, a s $X$ točka iz upita.
Indekse kod točaka u $N$-terokutu gledamo modulo $N$.  Za $20$ bodova bilo je
dovoljno znati kako provjeriti nalazi li se točka s neke strane pravca.
Funkcija koja daje odgovor na to popularno se zove $ccw$. Dakle, proći ćemo po
svim dužinama i ako za neki $i$ vrijedi $ccw(A_i, A_{i+1}, X) > 0$ i
$ccw(A_{i+\frac{n}{2}}, A_{i+\frac{n}{2}+1}, X) > 0$, to znači da se točka
nalazi u obojenom dijelu površine.

U dodatnih $30$ mogli ste implementirati takozvano \textit{offline} rješenje.
To je rješenje koje prvo učita sve upite, obradi u nekom poretku (ne nužno onom
u kojem su zadani u inputu) te onda redom odgovori na njih.

Za svih $110$ bodova bilo je potrebno osmisliti \textit{online} rješenje,
odnosno rješenje koje kada učita upit odmah odgovori na njega.

Ako je $ccw(A_{i}, A_{i+1}, X) > 0$, na $i$-tu stranicu $N$-terokuta ćemo
napisati $1$, a inače $0$. Sada je pitanje postoji li par nasuprotnih stranica
na kojima pise $1$.  Očito ne možemo provjeriti što piše na svakoj stranici jer
je to $\mathcal{O}(N)$, ali možemo nizom zamjedbi doći do efikasnijeg rješenja.

\textbf{Zamjedba 1}: Ako neke dvije stranice imaju $1$ na sebi, onda i sve
stranice između njih s neke strane $N$-terokuta imaju isto $1$ na sebi. Ovo
možemo lagano uočiti promatranjem stranica na kojima je napisana $0$.
Zamislimo da je točka iz upita izvor svjetlosti te da su stranice zidovi kroz
koje ne prodire svjetlost.  Možemo uočiti da, ako svjetlost može doći do zida
direktno iz izvora svijetlosti, onda je na tom zidu napisana $0$. Sada lako
vidimo da sve nule tvore interval na tom cikličnom nizu. Jasno, iz toga slijedi
da i jedinice također tvore interval.

\textbf{Zamjedba 2}: odgovor je \texttt{DA} ako i samo ako je interval jedinica
veličine barem $\frac{n}{2}+1$. Ovo je očito jer u $\frac{n}{2}+1$ uzastopnih
stranica moraju postojati dvije koje su nastuprotne.

Sada uzmimo proizvoljnu stranicu i njezinu nasuprotnu stranicu i pogledajmo
koje vrijednosti pišu na njima. Ako na objema piše $1$ možemo odmah ispisati
\texttt{DA}. Ako na objema pise $0$ možemo odmah ispisati \texttt{NE}. Ako na
jednoj stranici piše $1$, a na drugoj $0$ onda možemo prelomiti niz na dvije
polovice, svaka započinje stranicom za koju smo pogledali koja je vrijednost
gore. Jedna polovica je oblika \texttt{111...000}, a druga \texttt{000...111}.
U jednoj želimo pronaći zadnju jedinicu, u drugoj prvu jedinicu, a to možemo s
binarnom pretragom. Kad smo to pronašli zapravo smo pronašli i rubove našeg
intervala jedinica pa sad možemo lako odgovoriti na zadano pitanje.

Vremenska složenost $\mathcal{O}(q \log n)$
\end{document}
%%% Local Variables:
%%% mode: latex
%%% mode: flyspell
%%% ispell-local-dictionary: "croatian"
%%% End:
