%%%%%%%%%%%%%%%%%%%%%%%%%%%%%%%%%%%%%%%%%%%%%%%%%%%%%%%%%%%%%%%%%%%%%%
% Problem statement
\begin{statement}[
  problempoints=110,
  timelimit=2 sekunde,
  memorylimit=512 MiB,
]{Trobojnica}

  %\setlength\intextsep{-0.1cm}
%\begin{wrapfigure}[6]{r}{0.26\textwidth}
%\centering
%\includegraphics[width=0.26\textwidth]{img/flag.png}
%\end{wrapfigure}

,,\textit{Sve neka gori kad kroz vene crven, bijeli, plavi krene.}''
-- Slaven Bilić, 2008.

U ovom zadatku promatramo pravilne $N$-terokute kojima su stranice obojene u tri
boje, a vrhovi označeni prirodnim brojevima u smjeru kazaljke na satu.
\textit{Triangulacija} je podjela mnogokuta na trokute unutarnjim
dijagonalama takva da dijagonale nemaju zajedničkih točaka osim vrhova mnogokuta
te ne sijeku stranice mnogokuta osim u vrhovima mnogokuta. Naravno, u ovom
zadatku i svaka dijagonala mora biti obojena u jednu od tri boje.

Triangulacija je \textit{domoljubna} ako za svaki od $N-2$ trokuta vrijedi da su
mu sve tri stranice različite boje. Vaš je zadatak odrediti domoljubnu
triangulaciju zadanog mnogokuta.

%%%%%%%%%%%%%%%%%%%%%%%%%%%%%%%%%%%%%%%%%%%%%%%%%%%%%%%%%%%%%%%%%%%%%%
% Input
\subsection*{Ulazni podaci}
U prvom je retku prirodan broj $N$ iz teksta zadatka. \\
U drugom je retku $N$-teroznamenkasti broj čije znamenke predstavljaju boje
stranica $N$-terokuta u smjeru kazaljke na satu. Odnosno, prva znamenka
predstavlja boju stranice $(1,2)$, druga znamenka boju stranice $(2,3)$ i tako
sve do $N$-te znamenke koja predstavlja boju stranice $(N, 1)$. Dakako, boje su
označene znamenkama $1$, $2$ i $3$.

%%%%%%%%%%%%%%%%%%%%%%%%%%%%%%%%%%%%%%%%%%%%%%%%%%%%%%%%%%%%%%%%%%%%%%
% Output
\subsection*{Izlazni podaci}
Ako postoji domoljubna trijangulacija za zadani mnogokut, u prvi redak ispišite
riječ \texttt{DA}, a u protivnom ispišite riječ \texttt{NE}. Ako ste ispisali
\texttt{DA}, u svakom od sljedeća $N-3$ retka ispišite po jednu
dijagonalu u obliku $X$ $Y$ $C$, gdje su $X$ i $Y$ vrhovi dijagonale, a $C$ boja
$(1 \le X, Y \le N, 1 \le C \le 3)$. Ispisane dijagonale trebaju činiti
domoljubnu trijangulaciju ulaznog mnogokuta. Poredak dijagonala u ispisu nije
bitan. Ako postoji više domoljubnih triangulacija, ispišite bilo koju.

%%%%%%%%%%%%%%%%%%%%%%%%%%%%%%%%%%%%%%%%%%%%%%%%%%%%%%%%%%%%%%%%%%%%%%
% Scoring
\subsection*{Bodovanje}
{\renewcommand{\arraystretch}{1.4}
  \setlength{\tabcolsep}{6pt}
  \begin{tabular}{ccl}
 Podzadatak & Broj bodova & Ograničenja \\ \midrule
  1 & 20 & $4 \le N \le 11$ \\
  2 & 40 & $4 \le N \le 10^3$ \\
  3 & 50 & $4 \le N \le 2\cdot10^5$ \\
\end{tabular}}

Ako vaš program točno ispisuje prvi redak u svakom testnom primjeru nekog
podzadatka, osvojit će $10\%$ bodova predviđenih za taj podzadatak.

%%%%%%%%%%%%%%%%%%%%%%%%%%%%%%%%%%%%%%%%%%%%%%%%%%%%%%%%%%%%%%%%%%%%%%
% Examples
\subsection*{Probni primjeri}
\begin{tabularx}{\textwidth}{X'X'X}
\sampleinputs{test/trobojnica.dummy.in.1}{test/trobojnica.dummy.out.1} &
\sampleinputs{test/trobojnica.dummy.in.2}{test/trobojnica.dummy.out.2} &
\sampleinputs{test/trobojnica.dummy.in.3}{test/trobojnica.dummy.out.3}
\end{tabularx}

%\textbf{Pojašnjenje drugog probnog primjera:}

%\textbf{1. upit} \textrightarrow{}
%$A_9 = 9$, $A_{10} = 1 + 0 = 1$, $A_{11} = 1 + 1 = 2$,
%$A_{12} = 1 + 2 = 3$, $A_{13} = 1 + 3 = 4$.\\
%\phantom{\textbf{1. upit} \textrightarrow{}}
%$A_9 + A_{10} + A_{11} + A_{12} + A_{13} = 9 + 1 + 2 + 3 + 4 = 19$.

%\textbf{2. upit} \textrightarrow{}
%$A_{44} = 4 + 4 = 8$, $A_{45} = 4 + 5 = 9$. $A_{44} + A_{45} = 8 + 9 = 17$.

%%%%%%%%%%%%%%%%%%%%%%%%%%%%%%%%%%%%%%%%%%%%%%%%%%%%%%%%%%%%%%%%%%%%%%
% We're done
\end{statement}

%%% Local Variables:
%%% mode: latex
%%% mode: flyspell
%%% ispell-local-dictionary: "croatian"
%%% TeX-master: "../hio.tex"
%%% End:
