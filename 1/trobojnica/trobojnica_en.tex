%%%%%%%%%%%%%%%%%%%%%%%%%%%%%%%%%%%%%%%%%%%%%%%%%%%%%%%%%%%%%%%%%%%%%%
% Problem statement
\begin{statement}[
  problempoints=110,
  timelimit=1 sekunda,
  memorylimit=512 MiB,
]{Trobojnica}

  %\setlength\intextsep{-0.1cm}
%\begin{wrapfigure}[6]{r}{0.26\textwidth}
%\centering
%\includegraphics[width=0.26\textwidth]{img/flag.png}
%\end{wrapfigure}

%\subsection

,,\textit{Everything will be in flames once red, white and blue start running through your veins}''
-- Slaven Bilić

In this task, we will observe regular polygons that have each of their $N$ sides
colored in one of three colors and whose vertices are denoted from $1$ to $N$
in a clockwise order. \textit{Triangulaton} of a polygon is a decomposition of
its area into a set of non-intersecting triangles whose sides either lie on the
sides of the polygon or its internal diagonals. Of course, in this task, each of
the diagonals used for polygon triangulation is also colored in one of three
colors.

The triangulation is said to be \textit{patriotic} if each of its $N-2$
triangles has all three sides of different colors. Your task is to determine
a patriotic triangulation of a given polygon.

%%%%%%%%%%%%%%%%%%%%%%%%%%%%%%%%%%%%%%%%%%%%%%%%%%%%%%%%%%%%%%%%%%%%%%
% Input
\subsection*{Input}
The first line contains an integer $N$ from the task description. \\
The second line contains an integer consisting of $N$ digits which represent
the colors of polygon sides. More precisely, the first digit represents
the color of side $(1,2)$, the second one represents the color of side $(2,3)$,
and so on until the $N$-th digit which represents the color of side $(N,1$). The
colors will always be denoted with digits $1$, $2$ and $3$.

%%%%%%%%%%%%%%%%%%%%%%%%%%%%%%%%%%%%%%%%%%%%%%%%%%%%%%%%%%%%%%%%%%%%%%
% Output
\subsection*{Output}
If there is no patriotic triangulation of a given polygon, output \texttt{NE}
and terminate the program. Otherwise, in the first line output \texttt{DA}
and in the next $N-3$ lines output each diagonal used in the patriotic
triangulation. Each diagonal should be specified as $X$ $Y$ $C$ where $X$ and
$Y$ are the endpoints of a diagonal and $C$ is its color
$(1 \le X, Y \le N, 1 \le C \le 3)$. The order of diagonals in the output is
not important. If there are multiple correct solutions, output any of them.

%%%%%%%%%%%%%%%%%%%%%%%%%%%%%%%%%%%%%%%%%%%%%%%%%%%%%%%%%%%%%%%%%%%%%%
% Scoring
\subsection*{Scoring}
{\renewcommand{\arraystretch}{1.4}
  \setlength{\tabcolsep}{6pt}
  \begin{tabular}{ccl}
 Subtask & Score & Constraints \\ \midrule
  1 & 20 & $4 \le N \le 11$ \\
  2 & 40 & $4 \le N \le 10^3$ \\
  3 & 50 & $4 \le N \le 2\cdot10^5$ \\
\end{tabular}}

If your program correctly outputs the first line in each test case of a certain
subtask, you will score $10\%$ of the points allocated for that subtask.

%%%%%%%%%%%%%%%%%%%%%%%%%%%%%%%%%%%%%%%%%%%%%%%%%%%%%%%%%%%%%%%%%%%%%%
% Examples
\subsection*{Examples}
\begin{tabularx}{\textwidth}{X'X'X}
\sampleinputs{test/trobojnica.dummy.in.1}{test/trobojnica.dummy.out.1} &
\sampleinputs{test/trobojnica.dummy.in.2}{test/trobojnica.dummy.out.2} &
\sampleinputs{test/trobojnica.dummy.in.3}{test/trobojnica.dummy.out.3}
\end{tabularx}

%\textbf{Pojašnjenje drugog probnog primjera:}

%\textbf{1. upit} \textrightarrow{}
%$A_9 = 9$, $A_{10} = 1 + 0 = 1$, $A_{11} = 1 + 1 = 2$,
%$A_{12} = 1 + 2 = 3$, $A_{13} = 1 + 3 = 4$.\\
%\phantom{\textbf{1. upit} \textrightarrow{}}
%$A_9 + A_{10} + A_{11} + A_{12} + A_{13} = 9 + 1 + 2 + 3 + 4 = 19$.

%\textbf{2. upit} \textrightarrow{}
%$A_{44} = 4 + 4 = 8$, $A_{45} = 4 + 5 = 9$. $A_{44} + A_{45} = 8 + 9 = 17$.

%%%%%%%%%%%%%%%%%%%%%%%%%%%%%%%%%%%%%%%%%%%%%%%%%%%%%%%%%%%%%%%%%%%%%%
% We're done
\end{statement}

%%% Local Variables:
%%% mode: latex
%%% mode: flyspell
%%% ispell-local-dictionary: "croatian"
%%% TeX-master: "../hio.tex"
%%% End:
