%%%%%%%%%%%%%%%%%%%%%%%%%%%%%%%%%%%%%%%%%%%%%%%%%%%%%%%%%%%%%%%%%%%%%%
% Problem statement
\begin{statement}[
  problempoints=110,
  timelimit=1 second,
  memorylimit=512 MiB,
]{Trobojnica}

  %\setlength\intextsep{-0.1cm}
%\begin{wrapfigure}[6]{r}{0.26\textwidth}
%\centering
%\includegraphics[width=0.26\textwidth]{img/flag.png}
%\end{wrapfigure}

%\subsection

,,\textit{Everything will be in flames once red, white and blue start running through your veins}''
-- Slaven Bilić

Kako bi osvojili parcijalu od $10\%$ bodova, trebamo pronaći nužan i dovoljan uvjet postojanja
domoljubne triangulacije za mnogokut s obojanih stranica.

\textbf{Tvrdnja 1: } \; U svakoj triangulaciji postoji "uho", tj. trokut koji dijeli barem dvije
stranice s mnogokutom.

\emph{Skica dokaza: } \; Indukcija. Vrijedi za trokut.
Neka dijagonala dijeli mnogokut na dva manja mnogokuta, induktivno se pozovimo dalje. $\qed$

Neka na početku ima $a$, $b$ i $c$ stranica u bojama $1$, $2$ i $3$ redom.
Vrijedi $a+b+c = N$. Jasno je da mora biti $max\{a, b, c\} < n$, inače ne može postojati uho.

\textbf{Tvrdnja 2: } \; Ako postoji domoljubna triangulacija, brojevi $\{a, b, c\}$ su iste parnosti.

\emph{Skica dokaza: } \; Dvostruko prebrojavanje.
Prebrojimo li broj bridova boje $1$ po trokutima, dobivamo 

\begin{align*}
  2 * \text{# dijagonala boje 1} + a = N - 3,
\end{align*}

iz čega zaključujemo da je $a$ iste parnosti kao $N-3$. Analogno dobivamo isto i za $b$ i $c$. $\qed$

Navedeni uvjeti su zapravo dovoljni; to se najbolje dokazuje algoritmom koji je rješenje ovog zadatka.
Ukratko rečeno, induktivno ćemo raditi uha koristeći stranice dvije različite boje,
čime nećemo promijeniti (dokažite!) jednakost parnosti brojeva $\{a, b, c\}$ u novom mnogokutu, koji
ima jednu stranicu manje.
Jedino na što treba paziti je mogućnost da novi mnogokut ima sve stranice iste boje.

Jedna lijepa implementacija je sljedeće: za svaki vrh mnogokuta pamtimo njegovog sljedbenika u
trenutnom mnogokutu, te boju brida koji ih povezuje. Traženje mjesta na kojem možemo napraviti
uho zapravo nije veliki problem.
Naime, možemo samo ići u krug i napraviti uho na prvom vrhu čiji su prethodni i sljedbeni brid
iste boje, te je barem jedna od tih boja ona boja koje ima najviše među trenutnim stranicama.
Ukoliko svaki put krenemo od mjesta gdje smo napravili zadnje uho, moguće je pokazati da
je vremenska složenost amortizirani $O(N)$.

Uz opisano rješenje, razne nedeterminističke implementacije postižu dobre vremenske složenosti.
Postoje i kompliciranija $O(N log N)$ rješenja, koja također trebaju dobiti sve bodove.

\end{statement}

%%% Local Variables:
%%% mode: latex
%%% mode: flyspell
%%% ispell-local-dictionary: "croatian"
%%% TeX-master: "../hio.tex"
%%% End:
