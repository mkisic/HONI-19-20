%%%%%%%%%%%%%%%%%%%%%%%%%%%%%%%%%%%%%%%%%%%%%%%%%%%%%%%%%%%%%%%%%%%%%%
% Problem statement
\begin{statement}[
  problempoints=20,
  timelimit=1 sekunda,
  memorylimit=512 MiB,
]{Dinamo}


\setlength\intextsep{-0.1cm}
\begin{wrapfigure}[7]{r}{0.19\textwidth}
\centering
\includegraphics[width=0.19\textwidth]{img/chball.png}
\end{wrapfigure}

Godina je 2069., Dinamo slavi 50 godina svog prvog od ukupno deset osvajanja
Lige prvaka. Kile se prisjeća tog vremena i prvih 6 utakmica grupne faze
natjecanja. On se sjeća da je Dinamo u prvom kolu igrao protiv kluba s oznakom
$A$, u drugom protiv kluba $C$, a u trećem protiv $S$. Stari Kile se ne može
sjetiti s kim je Dinamo igrao u četvrtom, petom i šestom kolu.

Znamo da u Ligi prvaka vrijedi pravilo da u četvrtom kolu klub igra s
protivnikom s kojim je igrao u trećem kolu, u petom s protivnikom iz prvog kola,
a u šestom s onim iz drugog kola. Pomozi Kiletu i odgovori na njegovo pitanje
,,S kim smo ono igrali u $X$-tom kolu?''.

%%%%%%%%%%%%%%%%%%%%%%%%%%%%%%%%%%%%%%%%%%%%%%%%%%%%%%%%%%%%%%%%%%%%%%
% Input
\subsection*{Ulazni podaci}
U prvom je retku prirodan broj $A$ $(1 \le A \le 10)$ iz teksta zadatka. \\
U drugom je retku prirodan broj $C$ $(1 \le C \le 10)$ iz teksta zadatka. \\
U trećem je retku prirodan broj $S$ $(1 \le S \le 10)$ iz teksta zadatka. \\
U četvrtom je retku prirodan broj $X$ $(4 \le X \le 6)$ iz teksta zadatka.

Brojevi $A$, $C$ i $S$ međusobno su različiti.

%%%%%%%%%%%%%%%%%%%%%%%%%%%%%%%%%%%%%%%%%%%%%%%%%%%%%%%%%%%%%%%%%%%%%%
% Output
\subsection*{Izlazni podaci}
U jedini redak ispišite traženu oznaku kluba s kojim je Dinamo igrao u $X$-tom
kolu.

%%%%%%%%%%%%%%%%%%%%%%%%%%%%%%%%%%%%%%%%%%%%%%%%%%%%%%%%%%%%%%%%%%%%%%
% Scoring

%%%%%%%%%%%%%%%%%%%%%%%%%%%%%%%%%%%%%%%%%%%%%%%%%%%%%%%%%%%%%%%%%%%%%%
% Examples
\subsection*{Probni primjeri}
\begin{tabularx}{\textwidth}{X'X'X}
\sampleinputs{test/dinamo.dummy.in.1}{test/dinamo.dummy.out.1} &
\sampleinputs{test/dinamo.dummy.in.2}{test/dinamo.dummy.out.2} &
\sampleinputs{test/dinamo.dummy.in.3}{test/dinamo.dummy.out.3}
\end{tabularx}

\textbf{Pojašnjenje prvog probnog primjera:} \\
Dinamo je u prvom kolu igrao s timom koji ima oznaku $3$, u drugom s $5$, a u
trećem s timom $2$. U četvrtom kolu, prema pravilu iz teksta zadatka opet
je igrao s timom $2$.

%%%%%%%%%%%%%%%%%%%%%%%%%%%%%%%%%%%%%%%%%%%%%%%%%%%%%%%%%%%%%%%%%%%%%%
% We're done
\end{statement}

%%% Local Variables:
%%% mode: latex
%%% mode: flyspell
%%% ispell-local-dictionary: "croatian"
%%% TeX-master: "../hio.tex"
%%% End:
