\documentclass[a4paper]{article}
\usepackage{zadaci}
\usepackage{wrapfig}
\usepackage{url}
\usepackage{tikz}
\usepackage{amsmath}
\usepackage[normalem]{ulem}
\usetikzlibrary{angles,quotes}
\contestname{Hrvatsko otvoreno natjecanje u informatici\\6.\ kolo, 7. ožujka 2020.}
\markright{\textbf{\textsf{Opisi algoritama}}}

\usepackage{hyperref}
\hypersetup{
colorlinks=true,
linkcolor=blue,
filecolor=magenta,
urlcolor=cyan,
}

\begin{document}

\section*{Opisi algoritama}
Zadatke, testne primjere i rješenja pripremili: Fabijan Bošnjak, Nikola
Dmitrović, Karlo Franić, Marin Kišić, Ivan Paljak i Stjepan Požgaj.  Primjeri
implementiranih rješenja su dani u priloženim izvornim kodovima.

\subsection*{Zadatak: Ogledalo}
\textsf{Pripremio: Marin Kišić}\\
\textsf{Potrebno znanje: naredbe učitavanja i ispisivanja, rad s riječima}

Možemo primijetiti da će se uvijek unositi četiri riječi i da je potrebno
ispisati zadnju riječ bez zadnjeg znaka. Pola bodova moglo se osvojiti
ispisivanjem podriječi koja se proteže od 18.\ do 22.\ znaka rečenice iz ulaza.

\textit{Programski kod (pisan u \texttt{Python 3}):}

\vspace{-2ex}
\begin{verbatim}
print(input().split(' ')[-1][:-1])
\end{verbatim}

\subsection*{Zadatak: Ulica}
\textsf{Pripremio: Marin Kišić}\\
\textsf{Potrebno znanje: petlje, provjera parnosti}

Napravit ćemo pomoćno polje \texttt{moze} i za svaki broj $x$ iz ulaza ćemo
postaviti \texttt{moze[x]=1}. Nakon toga, prebrojimo ima li više parnih ili
neparnih brojeva u u ulazu. Na kraju, ako ima više parnih, krenemo od $2$ i
idemo tako dugo dok je \texttt{moze[trenutni\_broj]=1} krećući se po parnim
brojevima. Analogno za neparne, samo počinjemo od $1$.

\subsection*{Zadatak: Datum}
\textsf{Pripremio: Karlo Franić}\\
\textsf{Potrebno znanje: provjera palindromičnosti, ad-hoc}

Za prvi podzadatak potrebno je uzeti prva dva znaka datuma i taj broj
povećavati za $1$ dok ne nađemo palindrom.

Za drugi podzadatak koristimo znanje
prvog podzadatka, uz to svaki put kada dan dođe do zadnjeg u mjesecu (4.\ i 5.\
znak), postavimo dan na $1$, a mjesec povećamo za $1$.

Za treći podzadatak potrebno
je povećavati i godinu svaki put kad dođemo do $31$.\ dana u $12$.\ mjesecu.

Za sve
bodove bilo je potrebno primijetiti da palindromičnih datuma postoji $366$. Svaki
datum ima točno jednu godinu s kojom tvori palindrom. Te datume je moguće
staviti u polje i nakon toga za svaki zadani datum iterativno po polju naći
prvi datum koji je veći od zadanog.

Vremenska složenost je $\mathcal{O}(NK)$, gdje $K$ predstavlja broj
palindromičnih datuma. Zadatak se može riješiti i u složenosti $\mathcal{O}(N
\log K)$, a to rješenje ostavljamo čitateljici za vježbu.

\clearpage

\subsection*{Zadatak: Birmingham}
\textsf{Pripremio: Marin Kišić}\\
\textsf{Potrebno znanje: osnove teorije grafova, pretraga u širinu (BFS)}

Za pola bodova najprije trebamo odrediti polje \texttt{dist[a][b]} koje nam
govori najmanju udaljenost između čvorora $a$ i $b$. To možemo tako da iz
svakog čvora posbeno pustimo
\href{https://en.wikipedia.org/wiki/Breadth-first\_search}{BFS} prema svim
ostalim čvorovima. Nakon toga pustimo poseban BFS iz početnih čvorova koji se
širi na sljedeći način: ako smo trenutno u čvoru a s udaljenosti $d$, onda u
red ubacimo sve one čvorove koji još nisu posjećeni i koji su od $a$ udaljeni
za $\le (d-1) \cdot K$ koraka.  Njih nađemo tako da prođemo po nizu
\texttt{dist[a]}.

Za sve bodove bilo je potrebno uočiti da, ako znamo udaljenost čvora $x$ do
nekog od početnih čvorova, onda odmah možemo reći i koje je rješenje za njega
(formulom ili simulacijom, za implementacijske detalje pogledati službeno
rješenje ili rješenja natjecatelja koji su osvojili sve bodove). Udaljenosti do
početnih čvorova mogli smo odrediti tako da pustimo BFS iz početnih čvorova.

\subsection*{Zadatak: Konstrukcija}
\textsf{Pripremio: Stjepan Požgaj}\\
\textsf{Potrebno znanje: ad-hoc}

Moramo konstruirati graf s $N$ čvorova i $M$ bridova tako da je $tns(1, N) =
K$, gdje je $tns(x, y)=\sum_{C \in S_{x,y}}{sgn(C)}$. Neka je $[a, b>$ skup
čvorova $x$ takvih da postoji put od $a$ do $x$, postoji put od $x$ do $b$ i
$x$ je različit od $b$. Svakom uređenom nizu $C \in S_{x, y}$ možemo maknuti
zadnji član, te tako dobiti uređeni niz $C'$ koji počinje u $x$, završava u
nekom čvoru $[x, y>$ i čija je duljina za jedan manja od duljine uređenog niza
$C$ pa vrijedi $sgn(C) = -sgn(C')$. Zbog toga je $tns(x, y) = -\sum_{z \in [x,
y>}{tns(x,z)}$.

Naš graf gradit ćemo po razinama. Na prvoj razini će uvijek biti samo čvor $1$, a
na zadnjoj razini samo čvor $N$. U prva dva podzadatka će svi čvorovi (osim,
naravno, $N$) s neke razine imati usmjerene bridove prema svim čvorovima na
sljedećoj razini. U općenitom rješenju ćemo trebati malu modifikaciju.

Graf koji je rješenje za prvi podzadatak $1 \le K < 500$ imat će tri razine.
Na prvoj samo čvor s oznakom $1$, na drugoj $K + 1$ čvorova, a na zadnjoj čvor s
oznakom $K+3$. Primjerice, za $K=4$ graf izgleda kao na slici dolje.

\begin{figure}[!htbp]
\centering
\includegraphics[width=0.3\textwidth]{graf_1.png}
\end{figure}

U ovom primjeru, koristeći rekurzivnu relaciju za $tns(1, x)$, vidimo da je
$tns(1, 1) = 1$, $tns(1, 2) = \dots = tns(1, 6) = (-1) \cdot 1$, $tns(1, 7) =
(-1) \cdot (1 + 5 \cdot (-1)) = 4$.

Slično rješavamo i drugi podzadatak gdje vrijedi $-300 < K \le -1$. Primjerice,
za $K=-4$ graf izgleda kao na slici dolje.

\begin{figure}[!htbp]
\centering
\includegraphics[width=0.3\textwidth]{graf_2.png}
\end{figure}

Promatrajući konstrukcije za prva dva podzadatka možemo uočiti da, dodavanjem
$M$ čvorova na novu razinu, sumu svih vrijednosti $tns(1, x)$ množimo s $-(M -
1)$.  Također je iz rekurzivne relacije jasno da je $tns(1, N)$ jednak sumi
svih vrijednost $tns(1, x)$, gdje je $1 \le x < N$, pomnoženoj s $-1$. Ako je
broj $K$ oblika $\pm2^i$, dodavanjem tribrida na novu razinu, sumu vrijednosti
$tns(1, x)$ množimo s $-2$ pa tako s $3 + (i - 1) \cdot 9$ bridova možemo
dobiti vrijednost $K$, do na predznak.  Ako nam predznak nije dobar, na
sljedeću razinu dodajemo $2$ čvora da bismo sumu vrijednosti pomnožili s $-1$.
Sad još moramo odrediti kako apsolutnu vrijednost trenutne sume vrijednosti
povećati za $1$. Pomoću toga možemo konstruirati graf algoritmom analognim onom
u kojem množenjem s $2$ i eventualnim dodavanjem jedinice iz broja u binarnom
zapisu dobivamo njegovu vrijednost u dekadskom zapisu. Ako je trenutna suma
vrijednosti negativna, jednostavno joj možemo oduzeti $1$ tako da na trenutnu
razinu dodamo čvor koji je spojen samo sa čvorom $1$. Isto tako, ako je ona
pozitivna, možemo dodati $2$ čvora na novu razinu spojena sa svima s prethodne
da tu vrijednost učinimo negativnom te tada na isti način kao u prethodnom
slučaju oduzeti $1$.

Ovako opisan algoritam za dani $K$ konstruira graf s manje od $16 \cdot
\log_2(K)$ bridova.

\subsection*{Zadatak: Skandi}
\textsf{Pripremili: Fabijan Bošnjak i Ivan Paljak}\\
\textsf{Potrebno znanje: maksimalno sparivanje u bipartitnom grafu, minimalni
vršni pokrivač, K\"{o}nigov teorem}

Za osvajanje bodova na prvom podzadatku bilo je dovoljno primijetiti da je,
zbog malog broja jedinica, broj pitanja u skandinavci vrlo malen. Bilo je
dovoljno na sve moguće načine pretpostaviti na koja ćemo pitanja ponuditi
odgovor te potom, za svaki način, provjeriti je li cijela skandinavka ispunjena.
Vremenska složenost je $\mathcal{O}(2^QNM)$ gdje sa $Q$ označavamo ukupan broj
pitanja u skandinavci.

Ograničenja drugog podzadatka će iskusnog natjecatelja odmah navesti na pravi
put. Matrica je
\href{https://blogaritam.com/2018/09/10/tema-dana-duguljaste-matrice/}{duguljasta},
naziru se bitmaske, mora da se radi o dinamičkom programiranju. I bi tako,
kretat ćemo se matricom ,,redak po redak'' te ćemo pamtiti u kojim smo stupcima
odlučili odgovoriti na okomita pitanja čiji se odgovori protežu kroz trenutni
redak.  Dakle, stanje označavamo sa \texttt{dp[red][maska]} koje nam govori na
koliko najmanje pitanja je potrebno odgovoriti ako smo popunili sve retke
skandinavke do retka $red - 1$, nalazimo se u retku $red$, a u masci su
navedeni nezavršeni odgovori na okomita pitanja. Razradu prijelaza kao i
rekonstrukciju ostavit ćemo čitateljici za vježbu. Implementaciju ovog rješenja
vremenske složenosti $\mathcal{O}(N2^M)$ možete vidjeti u izvornom kodu
\texttt{skandi\_dp.cpp}.

Ponekad je korisno formalnije zapisati ono što se od nas u zadatku traži. U
ovom slučaju, taj će nas pristup prirodno odvesti do potpunog rješenja.
Označimo \textit{vodoravna} pitanja sa $a_1, a_2, \dots, a_p$, a okomita
pitanja sa $b_1, b_2, \dots, b_q$. Neka je $a_i$ (odnosno $b_j$) jednak $1$ ako
smo na to pitanje ponudili odgovor, odnosno neka je jednak $0$ ako to nismo
napravili. Primijetimo da za svako prazno polje $x$ moramo odabrati barem jedno
od točno dva pitanja u čijem se odgovoru to polje nalazi. Dakle, za svako
prazno polje postoje neka dva pitanja $a_i$ i $b_j$ za koja mora vrijediti $a_i
\lor b_j = 1$.

Modelirajmo sada ovaj problem bipartitnim grafom u kojem se s jedne strane
nalaze čvorovi koji predstavljaju vodoravna, a s druge strane čvorovi koji
predstavljaju okomita pitanja. Povežimo neka dva čvora $a_i$ i $b_j$ bridom ako
postoji neko prazno polje koje uvjetuje da vrijedi $a_i \lor b_j = 1$.  Ono što
želimo napraviti jest odrediti najmanji mogući skup čvorova takav da je svaki
brid u grafu incidentan (susjedan) barem jednom čvoru iz tog skupa.  Dakako,
radi se poznatom problemu traženja minimalnog vršnog pokrivača (\textit{engl.}\
\href{https://en.wikipedia.org/wiki/Vertex\_cover}{minimum vertex cover}).
Budući da je graf bipartitan, možemo iskoristiti
\href{https://en.wikipedia.org/wiki/K%C5%91nig%27s_theorem_(graph_theory)}{K\"{o}nigov
teorem} i zadatak riješiti u polinomijalnoj složenosti.

\subsection*{Zadatak: Trener}
\textsf{Pripremio: Karlo Franić}\\
\textsf{Potrebno znanje: teorija grafova, dinamičko programiranje}

Za prvi podzadatak potrebno je provjeriti svaku mogućnost. Složenost je
$\mathcal{O}(K^N)$.

Za druga dva podzadatka bilo je potrebano izgraditi acikličan usmjereni graf
(DAG) i na njemu prebrojiti broj puteva od čvorova s prve razine do čvorova sa
zadnje razine. To prebrojavanje može se rješiti dinamičkim programiranjem i
prepuštamo ga čitateljici za vježbu. Objasnimo sada kakav točno DAG želimo
izgraditi.

Na svakoj od $N$ razina imat ćemo čvor za svako različito prezime koje se
pojavljuje na toj razini. Vrijednost čvora bit će broj istih prezimena koje on
predstavlja. Veze ćemo graditi od čvorova na razini i prema čvorovima na razini
$i+1$. Veza od $A$ prema $B$ će postojati ako prezime koje predstavlja čvor $A$
se razlikuje u točno jednom slovu od prezima koje predstavlja čvor $B$. U
drugom podzataku te veze smo mogli ručno izraditi, međutim za sve bodove bilo
je potrebno korištenje efikasnih ugrađenih struktura podataka (npr.\
\texttt{std::map} ili \texttt{std::unordered\_map} u jeziku C++) ili ručno
hashiranje podataka.

\end{document}
%%% Local Variables:
%%% mode: latex
%%% mode: flyspell
%%% ispell-local-dictionary: "croatian"
%%% End:
