%%%%%%%%%%%%%%%%%%%%%%%%%%%%%%%%%%%%%%%%%%%%%%%%%%%%%%%%%%%%%%%%%%%%%%
% Problem statement
\begin{statement}[
  problempoints=50,
  timelimit=1 sekunda,
  memorylimit=512 MiB,
]{Datum}

\setlength\intextsep{-0.1cm}
\begin{wrapfigure}[6]{r}{0.22\textwidth}
\centering
\includegraphics[width=0.22\textwidth]{img/datum.png}
\end{wrapfigure}

Sezona ispita je gotova te većina studenata Fakulteta elektrotehnike i
računarstva zasluženo provodi slobodno vrijeme spavajući. U rijetkim
trenucima kada su budni, uzimaju mobitel u ruke i vrijeme provode gledajući
nove objave na Instagramu. Fabijan je jedan od tih studenata.

Nedavno je naišao na sljedeću objavu -- datum \texttt{02.02.2020.} je prvi
palindromični datum u posljednjih $909$ godina.

 Ta ga je objava navela na razmišljanje o značenju izraza \textit{palindromični
datum} te se za $N$ datuma zapitao koji je prvi sljedeći palindromični datum
nakon njega. Datum je \textit{palindromičan} ako se, zanemarivši točke,
slijeva nadesno čita jednako kao i zdesna nalijevo. Primjerice, datumi
\texttt{02.02.2020.} i \texttt{12.10.0121.} su palindromični, dok
\texttt{03.02.2020.} i \texttt{12.07.1993.} to nisu.

\textbf{Napomena}: u zadatku je potrebno obratiti pozornost na prijestupne
godine koje u veljači imaju $29$ dana. Za potrebe ovog zadatka, smatramo da je
godina prijestupna ako je djeljiva s $4$.

%%%%%%%%%%%%%%%%%%%%%%%%%%%%%%%%%%%%%%%%%%%%%%%%%%%%%%%%%%%%%%%%%%%%%%
% Input
\subsection*{Ulazni podaci}
U prvom je retku prirodan broj $N$ $(1 \le N \le 10\ 000)$ iz teksta zadatka.\\
U sljedećih se $N$ redaka nalazi datum u formatu \texttt{DD.MM.YYYY.}

%%%%%%%%%%%%%%%%%%%%%%%%%%%%%%%%%%%%%%%%%%%%%%%%%%%%%%%%%%%%%%%%%%%%%%
% Output
\subsection*{Izlazni podaci}
Za svaki datum iz ulaza ispišite prvi sljedeći palindromični datum. Traženi
odgovor je potrebeno (te će uvijek biti moguće) ispisati u formatu
\texttt{DD.MM.YYYY.}

%%%%%%%%%%%%%%%%%%%%%%%%%%%%%%%%%%%%%%%%%%%%%%%%%%%%%%%%%%%%%%%%%%%%%%
% Scoring
\subsection*{Bodovanje}
U testnim primjerima vrijednima $10$ bodova, vrijedit će $N=10$ te će svaki traženi
datum biti unutar istog mjeseca i godine kao i zadani datum.

U testnim primjerima vrijednima dodatnih $10$ bodova, vrijedit će $N=10$ te će svaki
traženi datum biti unutar iste godine kao i zadani datum.

U testnim primjerima vrijednima dodatnih $20$ bodova vrijedit će $N=10$.

%%%%%%%%%%%%%%%%%%%%%%%%%%%%%%%%%%%%%%%%%%%%%%%%%%%%%%%%%%%%%%%%%%%%%%
% Examples
\subsection*{Probni primjeri}
\begin{tabularx}{\textwidth}{X'X'X}
\sampleinputs{test/datum.dummy.in.1}{test/datum.dummy.out.1} &
\sampleinputs{test/datum.dummy.in.2}{test/datum.dummy.out.2} &
\sampleinputs{test/datum.dummy.in.3}{test/datum.dummy.out.3}
\end{tabularx}

\textbf{Pojašnjenje prvog probnog primjera:}
ako je zadani datum palindromičan, Fabijana zanima prvi sljedeći koji je
palindromičan, a to je upravo \texttt{12.02.2021.}


%%%%%%%%%%%%%%%%%%%%%%%%%%%%%%%%%%%%%%%%%%%%%%%%%%%%%%%%%%%%%%%%%%%%%%
% We're done
\end{statement}

%%% Local Variables:
%%% mode: latex
%%% mode: flyspell
%%% ispell-local-dictionary: "croatian"
%%% TeX-master: "../hio.tex"
%%% End:
