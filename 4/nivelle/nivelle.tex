%%%%%%%%%%%%%%%%%%%%%%%%%%%%%%%%%%%%%%%%%%%%%%%%%%%%%%%%%%%%%%%%%%%%%%
% Problem statement
\begin{statement}[
  problempoints=110,
  timelimit=1 sekunda,
  memorylimit=512 MiB,
]{Nivelle}

\setlength\intextsep{-0.1cm}
\begin{wrapfigure}[9]{r}{0.17\textwidth}
\centering
\includegraphics[width=0.17\textwidth]{img/nivelle.png}
\end{wrapfigure}

\textit{Originalni tekst ovog zadatka zamijenjen je zbog pretjeranog nasilja.
Slijedi program primjeren djeci i maloljetnicima.}

Bojan vidi $N$ malih slatkih plišanaca \textit{(jeej!)} poredanih na polici u
dućanu, redom na mjestima $1$, $2$, \dots, $N$. Svaki pliš plišanac jedne je od
$26$ različitih boja. Boje označavamo malim slovima engleske abecede. Bojan
želi pojesti neke od tih plišanaca \textit{(njam!)}.

Za bilo koji skup plišanaca možemo definirati \textit{šarenilo} kao broj
različitih boja plišanaca u skupu, podijeljen s ukupnim brojem plišanaca u
skupu. Bojanu šarenilo bljak. Bojan jako gladan. Hoće papati uzastopni podniz
plišanaca.

Pomozite Bojanu da njam njam uzastopni podniz plišanaca najmanjeg mogućeg
šarenila.

%%%%%%%%%%%%%%%%%%%%%%%%%%%%%%%%%%%%%%%%%%%%%%%%%%%%%%%%%%%%%%%%%%%%%%
% Input
\subsection*{Ulazni podaci}
U prvom je retku prirodan broj $N$ $(1 \le N \le 100\ 000)$, duljina niza
plišanaca iz teksta zadatka.

U drugom je retku niz malih slova engleske abecede $S$ duljine $N$, redom boje
plišanaca iz teksta zadatka.

%%%%%%%%%%%%%%%%%%%%%%%%%%%%%%%%%%%%%%%%%%%%%%%%%%%%%%%%%%%%%%%%%%%%%%
% Output
\subsection*{Izlazni podaci}
Ispišite dva indeksa $L$ i $R$ $(1 \le L \le R \le N)$, koji označavaju da se
traženi uzastopni podniz plišanaca nalazi na mjestima $L$, $L+1$, \dots, $R$.

Ako postoji više podnizova iste tražene vrijednosti, ispišite $L$ i $R$ koji
odgovaraju bilo kojemu.

%%%%%%%%%%%%%%%%%%%%%%%%%%%%%%%%%%%%%%%%%%%%%%%%%%%%%%%%%%%%%%%%%%%%%%
% Scoring
 \subsection*{Bodovanje}
{\renewcommand{\arraystretch}{1.4}
  \setlength{\tabcolsep}{6pt}
  \begin{tabular}{ccl}
 Podzadatak & Broj bodova & Ograničenja \\ \midrule
  1 & 7 & $N \le 100$ \\
  2 & 17 & $N \le 2\ 000$ \\
  3 & 13 & $S$ se sastoji samo od slova \texttt{'a'} i \texttt{'b'} \\
  4 & 25 & $S$ se sastoji samo od slova  \texttt{'a'}, \texttt{'b'},
               \texttt{'c'}, \texttt{'d'} i \texttt{'e'} \\
  5 & 48 & Nema dodatnih ograničenja.
\end{tabular}}

%%%%%%%%%%%%%%%%%%%%%%%%%%%%%%%%%%%%%%%%%%%%%%%%%%%%%%%%%%%%%%%%%%%%%%
% Examples
\subsection*{Probni primjeri}
\begin{tabularx}{\textwidth}{X'X'X}
\sampleinputs{test/nivelle.dummy.in.1}{test/nivelle.dummy.out.1} &
\sampleinputs{test/nivelle.dummy.in.2}{test/nivelle.dummy.out.2} &
\sampleinputs{test/nivelle.dummy.in.3}{test/nivelle.dummy.out.3}
\end{tabularx}

%%%%%%%%%%%%%%%%%%%%%%%%%%%%%%%%%%%%%%%%%%%%%%%%%%%%%%%%%%%%%%%%%%%%%%
% We're done
\end{statement}

%%% Local Variables:
%%% mode: latex
%%% mode: flyspell
%%% ispell-local-dictionary: "croatian"
%%% TeX-master: "../hio.tex"
%%% End:
