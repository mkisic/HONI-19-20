%%%%%%%%%%%%%%%%%%%%%%%%%%%%%%%%%%%%%%%%%%%%%%%%%%%%%%%%%%%%%%%%%%%%%%
% Problem statement
\begin{statement}[
  problempoints=20,
  timelimit=1 sekunda,
  memorylimit=512 MiB,
]{FPS}

\setlength\intextsep{-0.1cm}
\begin{wrapfigure}[7]{r}{0.26\textwidth}
\centering
\includegraphics[width=0.26\textwidth]{img/fps.png}
\end{wrapfigure}

Naši dragi prijatelji Fabijan i Patrik su prošle godine bili jako dobri te su
za Božić zaslužili odlične poklone. Fabijan je od Djeda Mraza tražio dva
kontrolera, a njegov prijatelj Patrik najbolju igru na svijetu. Naravno, ta
igra je \textit{FIFA 20}. Na Božićno su jutro bili oduševljeni jer su pod
drvcem našli upravo ono što su tražili. Prepuni veselja našli su se kod
Patrika kako bi što prije započeli s igranjem. Za svoju prvu utakmicu
odabrali su okršaj titana s dna HNL tablice, Fabijan će upravljati igračima
Varaždina, a Patrik će igrati protiv njega u dresovima Istre.

Kako bi stigli na Božićni ručak, u postavkama igre su postavili da utakmica
traje točno $X$ minuta. Patrik na svom računalu može igrati igru u $Y$
FPS-a (\textit{engl.}\ Frames Per Second (sličica u sekundi)), tj. njegovo
računalo svake sekunde prikaže $Y$ sličica. Nakon što su odigrali utakmicu
  Fabijan je pitao Patrika: \textit{,,Patriče, koliko je sličica tvoje računalo
  prikazalo za vrijeme ove utakmice?''}. Dečki se za vrijeme praznika ne žele
baviti matematikom pa vas mole da odgovorite na Fabijanov upit.


%%%%%%%%%%%%%%%%%%%%%%%%%%%%%%%%%%%%%%%%%%%%%%%%%%%%%%%%%%%%%%%%%%%%%%
% Input
\subsection*{Ulazni podaci}
U prvom je retku prirodan broj $X$ $(1 \le X \le 100)$ iz teksta zadatka.

U drugom je retku prirodan broj $Y$ $(1 \le Y \le 100)$ iz teksta zadatka.

%%%%%%%%%%%%%%%%%%%%%%%%%%%%%%%%%%%%%%%%%%%%%%%%%%%%%%%%%%%%%%%%%%%%%%
% Output
\subsection*{Izlazni podaci}
U jedini redak ispišite broj sličica prikazanih na Patrikovom računalu za
vrijeme utakmice.

%%%%%%%%%%%%%%%%%%%%%%%%%%%%%%%%%%%%%%%%%%%%%%%%%%%%%%%%%%%%%%%%%%%%%%
% Scoring

%%%%%%%%%%%%%%%%%%%%%%%%%%%%%%%%%%%%%%%%%%%%%%%%%%%%%%%%%%%%%%%%%%%%%%
% Examples
\subsection*{Probni primjeri}
\begin{tabularx}{\textwidth}{X'X'X}
\sampleinputs{test/fps.dummy.in.1}{test/fps.dummy.out.1} &
\sampleinputs{test/fps.dummy.in.2}{test/fps.dummy.out.2} &
\sampleinputs{test/fps.dummy.in.3}{test/fps.dummy.out.3}
\end{tabularx}

\textbf{Pojašnjenje prvog probnog primjera:}
Jedna minuta sadrži $60$ sekundi. Ako se svake sekunde prikaže jedna sličica,
ukupno će biti prikazano $60$ sličica.

%%%%%%%%%%%%%%%%%%%%%%%%%%%%%%%%%%%%%%%%%%%%%%%%%%%%%%%%%%%%%%%%%%%%%%
% We're done
\end{statement}

%%% Local Variables:
%%% mode: latex
%%% mode: flyspell
%%% ispell-local-dictionary: "croatian"
%%% TeX-master: "../hio.tex"
%%% End:
