\documentclass[a4paper]{article}
\usepackage{zadaci}
\usepackage{wrapfig}
\usepackage{url}
\usepackage{tikz}
\usepackage{amsmath}
\usepackage[normalem]{ulem}
\usetikzlibrary{angles,quotes}
\contestname{Croatian Open Competition in Informatics\\Round 3, December 14\textsuperscript{th} 2019}
\markright{\textbf{\textsf{Editorial}}}

\usepackage{hyperref}
\hypersetup{
colorlinks=true,
linkcolor=blue,
filecolor=magenta,
urlcolor=cyan,
}

\begin{document}

\section*{Editorial}
Tasks, test data and solutions were prepared by: Fabijan Bošnjak, Nikola
Dmitrović, Marin Kišić, Josip Klepec, Daniel Paleka, Ivan Paljak, Tonko
Sabolčec and Paula Vidas. Implementation examples are given
in attached source code files.

\subsection*{Task: Preokret}
\textsf{Suggested by: Nikola Dmitrović}\\
\textsf{Necessary skills: loops, arrays}

This task is made for beginners and consists of three parts, each slightly
more difficult than the last.

In the first part you were asked to determine what was the final score of the
game. It was enough to simply count how many times did the numbers $1$ and $2$
occur in the input. This can be done simply by looping through the input data
and storing that information in two variables.

In the second part you were asked to determine the number of ties. Since you
already know how many goals each team scored in every moment (from part 1),
you can simply introduce another variable which should be increased whenever
the number of scored goals was the same for both teams. You also needed to
make sure to include the starting score (\texttt{0:0}).

In the third and final part you were asked to determine what was the largest
turnover. Note that each turnover consists of three parts. First, one team is
losing and starts to score goals, then the score comes to a tie and finally that
team starts winning. Suppose we store the difference in goals between the two
teams. In that case, we are simply looking for the largest interval in which
the difference at the start of the interval was negative and it became positive
at the end (or vice versa).

\subsection*{Task: Grudanje}
\textsf{Suggested by: Marin Kišić}\\
\textsf{Necessary skills: prefix sums, binary search}

In order to score $14$ points, it was enough to simply simulate the process
described in the problem statement.

The time complexity of that solution is $\mathcal{O}(NQN)=\mathcal{O}(QN^2)$.

In order to score additional $14$ points, you could use prefix sums to speed
up the solution above. Suppose that for every letter you built an array $pref$
in which an element at position $i$ stores the number of occurences of that
letter from the beginning of the string to the $i$-th letter (inclusive). For
example, for a word \texttt{abcaab} and a latter \texttt{'a'}, the array $pref$
would be $[1, 1, 1, 2, 3, 3]$. That array is useful because now we can determine
in $\mathcal{O}(1)$ what is the number of occurrences of a certain letter in
an interval from $L$-th to the $R$-th letter in our original word. The formula
for that is $pref[R]-pref[L-1]$.

The idea is now to precompute this array for every letter, thus obtaining an
algorithm with complexity $\mathcal{O}(N(N\Sigma * Q\Sigma))=
\mathcal{O}((N^2 + Q)\Sigma)$, where $\Sigma$ denotes the size of our
alphabet.

In order to score all points, you could observe the following: if a word is
prefect after $i$-th snowball, it will also be perfect after every other snowball
thrown after the $i$-th one. This property allows us to use binary search. The
check inside binary search is done similarly to the algorithm above (using
prefix sums).

Time complexity: $\mathcal{O}((N\Sigma + Q\Sigma)\log N) = \mathcal{O}((N \log N + Q)\Sigma)$.

\clearpage

\subsection*{Task: Drvca}
\textsf{Suggested by: Marin Kišić i Josip Klepec}\\
\textsf{Necessary skills: ad-hoc, basic data structures}

More formally, the task statement asks us to divide the input data into
two arithmetic sequences.

To obtain $20$ points, it was enough to try all possible combinations and
check whether some of them satisfies all conditions. The time complexity
of this solution is $\mathcal{O}(N \cdot 2^N)$.

For additional $30$ points, it was enough to observe that the smallest
number in the input must also be the smallest number in one of the
arithmetic sequences. Now, let's fix the second smallest number in that
sequence. Now we have also fixed the difference between two successive
elements of that sequence, meaning that we can easily calculate what number
could be next in the sequence. We will add new numbers to that sequence,
one by one, and each time check whether the remaining numbers form another
arithmetic sequence. The time complexity of this solution is $\mathcal{O}(N^3)$.

As Bob the Builder would say: ``Can we fix it? -- Yes, we can!``

We can note that the two smallest numbers of one arithmetic sequence could only
be fixed in three different ways. If we observe the three smallest numbers in
the input (denoting them with $A \le B \le C$), we can easily conclude that one
arithmetic sequence must start with either $(A, B)$, $(A, C)$ or $(B, C)$.

This observation leads us to an algorithm of time complexity $\mathcal{O}(N^2)$.

Finally, to obtain all points, we can speed up the idea from above. More
precisely, we will speed up the check whether all elements that were
not included in first sequence form another arithmetic sequence. We will
keep two (multi)sets around. The first set will contain all remaining numbers, and
the other will contain the differences of neighbouring elements from first
set. When we remove the number from the first set, i.e. append it to the
first sequence, we need to remove the difference between it and its neighbours
from the second set. We also need to add the difference between the neighbours
of removed element because they have now become neighbouring elements. The check
whether the remaining number form an arithmetic sequence boils down to checking
whether the smallest element from the second set is equal to the largest element
in that same set. All operations on the aforementioned sets are supported
by the STL collection \texttt{std::set} in C++. Naturally, other supported
languages (with exception of C) also offer similar built-in collections.

The time complexity is $\mathcal{O}(N \log N)$.

\subsection*{Task: Lampice}
\textsf{Suggested by: Tonko Sabolčec}\\
\textsf{Necessary skills: hashing, binary search, centroid decomposition of a
tree}

The first subtask can be solved in multiple ways. We will describe the solution
using hashing which will be useful for the final subtask. Let's root the tree
in a node $R$. We are interested in all palindromic segments with one end in
node $R$. For every node $X$, we will calculate two values:

\[down_x = x_0B^{k-1} + x_1B^{k-2} + \dots + x_{k-1}B^0\]
\[up_x = x_0B^0 + x_1B^1 + \dots + x_{k-1}B^{k-1}\]

Where $x_0$, $x_1$, \dots, $x_{k-1}$ is an array of colors on the path from
$R$ to $X$ ($color(R) = x_0$, $color(X) = x_{k-1}$) and $B$ is the value
of the hash basis. These two values can be determined using a single DFS
traversal of our tree. The path from $R$ to $X$ is a palindromic segment if
$down_X = up_X$ holds. If we root the tree in every node and apply this
procedure, we will cover all cases. The complexity of this algorithm is
$\mathcal{O}(N^2)$.

\clearpage

The second subtask is a classic, well-known problem of finding the longest
palindromic substring which can be solved using
\href{https://en.wikipedia.org/wiki/Longest_palindromic_substring}{Manacher's
algorithm}.

The solution for the third subtask is quite similar to the solution of the
second subtask. The only difference is that we must apply Manacher's algorithm
on a path between every pair of leaves.

We will begin our solution for the whole problem with the following observation:
If there exists a palindromic segment of length $K > 2$, then there exists a
palindromic segment of length $K - 2$. Therefore, we can use binary search on
even and odd lengths to find the longest palindromic segment.

But, how do we check if a palindromic segment of fixed length exists in a
given tree? To answer that question, we will first solve an easier version
of that problem -- we will check if there is a palindromic segment of a fixed
length that contains the root of our tree, node $R$.

In a rooted tree, a path between two nodes $A$ and $B$ consists of two
tree branches, where the length of path on one branch is greater or equal
to the length on the other branch. We will divide the path between $A$ and
$B$ into three parts: path from $A$ to $C$, path from $C$ to the root and path
from the root to $C$, where $C$ is chosen such that the lengths $A-C$ and $R-B$
are equal. Note that $C$ is uniquely defined by the node $A$, since we only care
about palindromes of fixed size. In order to check whether the path from $A$ to
$B$ is a palindromic segment, it is enough to check the following:

\begin{itemize}
  \item (1) Path from $R$ to $C$ is a palindrome (depicted in red)
  \item (2) Colors from $C$ to $A$ are the same as colors from $R$
    to $B$ (depicted blue).
\end{itemize}

\begin{figure}[!htbp]
\centering
\includegraphics[width=0.4\textwidth]{pic/opisi_lampice.png}
\end{figure}

For each node of the tree we will calculate the values $down_X$ and $up_X$ in
advance in the same manner as it was described in the solution of the first
subtask. The first check is simply done by comparing $down_C$ and $up_C$. The
second check can be done by comparing the values:

\begin{itemize}
  \item $down_B$
  \item (3) $down_A - down_{par(C)} \cdot B^{dep(A)-dep(C)}$, where $par(C)$ is
        the parent of node $C$ and $dep(X)$ is the depth of node $X$.
\end{itemize}

Now we know how to efficiently check whether path from $A$ to $B$ is a
palindromic segment, but that is still not enough because there are $\mathcal{O}(N^2)$
such pairs $(A, B)$. Let's try to observe the problem from another angle: If we
fix $A$, is there (at least) one $B$ that satisfies all the conditions?. Let
$S_B$ be a set of hashes in which we will insert values $down_B$ of all nodes
$B$ of our tree. Then, for a fixed $A$, we can simply check condition (1) and
we can check condition (3) by finding the corresponding value in $S_B$. We can
assume that the complexity of all operations with $S_B$ is constant if we use
a hash table (\texttt{std::unordered\_set} in C++, for example). The complexity
of the whole check is therefore $\mathcal{O}(n)$.

\textbf{Note:} during implementation you need to make sure that the lowest
common ancestor of of nodes in $S_B$ and $A$ is the root node $R$.

Now that we know whether a palindromic segment of certain length which passes
through the root $R$ exists, we can use centroid decomposition. If we perform
this check for every decomposed subtree with the centroid as a root, we will
cover all the cases. With centroid decomposition and a binary search, we end up
with an algorithm of complexity  $\mathcal{O}(n \log^2 n)$.

\subsection*{Task: Sob}
\textsf{Suggested by: Paula Vidas i Daniel Paleka}\\
\textsf{Necessary skills: matematika, pohlepni algoritmi}

Let's consider an ordered pair $(a, b)$ to be \emph{good} if $a \mathbin\& b = a$.

We can solve the first subtask by matching $a \in A$ with $b \in B$ for which
$b \mathbin{\textrm{mod}} N = a$ holds.

The second subtask can be solved with the following algorithm. Let
$i_1 > i_2 > ... > i_k$ be the positions of ones in a binary notation
of $N$. We will match the smallest $2^{i_1}$ elements of $A$ and $B$ such that
we match those $a$ and $b$ for which $a \equiv b \mod 2^{i_1}$ holds. Then we
take the following $2^{i_2}$ elements and match those that are equal modulo
$2^{i_2}$, and so on. The proof of correctness is left as an exercise for the
reader.

The third subtask can be solved in multiple ways. One of those ways is to
build a bipartite graph in which each node correspond to one member of the
sets $A$ and $B$. Naturally, we add edges between all good pairs of nodes. The
problem now boils down \emph{bipartite matching} which could have been
implemented by a standard $\mathcal{O}(NE)$ algorithm, where $E$ denotes the
number of edges and $N$ denotes the number of nodes. Note that we can deduce
the upper bound on $E < 3^{10} = 59049$.

The other way is by using the following greedy algorithm. We will traverse
through the elements of $A$ from largest to smallest and we will match the
current element with the smallest unmatched element from $B$ which forms a
good pair with our current element in $A$.

If we run this algorithm on a couple of examples, we can observe the following.
Suppose that the largest element of $A$, i.e. $N - 1$, is matched with $b \in B$.
Then, $N - 1 - t$ will also be matched with $b - t$ for each
$t \in \{1, 2, ..., b - M\}$. After we remove those matched pairs, we are left
with the same problem on smaller sets $A' = \{0, 1, ..., N - 1 - (b - M) - 1\}$
and $B' = \{b + 1, b + 2, ..., M + N - 1\}$. This solution can easily be
implemented in $\mathcal{O}(N)$.

Proof of the former observation:\\
Let $a = N - 1$ and let $b$, as before, be the smallest element of $B$ for which
$a \mathbin\& b = a$. With index $i$ we will denote the $i$-th digit of weight
$2^i$. If $b = M$ there is nothing left to prove, so we will assume that
$b > M$ and introduce $k = b - M$. Let $i$ be the position of the smallest
significant one in $b$. Obviously it must hold that $a_j = b_j = 0$ for all
$j < i$. If $a_i == 0$, then $a \mathbin\& (b - 1) = a$ would hold and $b$
wouldn't be the smallest element in $B$ that can be matched with $a$. Therefore,
$a_i = b_i = 1$. Now it's obvious that $(a - t, b - t)$ is a good pair for
$t \in \{1, 2, ..., 2^i\}$. If $k \leq 2^i$, we are done, otherwise we observe
the next smallest significant one in $b$ and go through the same procedure
inductively. The only thing left to prove is that there is always a $b \in B$
which can be matched with $a$, but we will leave this part of the proof as
an exercise for the reader.

\end{document}
%%% Local Variables:
%%% mode: latex
%%% mode: flyspell
%%% ispell-local-dictionary: "english"
%%% End:

