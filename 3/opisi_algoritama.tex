\documentclass[a4paper]{article}
\usepackage{zadaci}
\usepackage{wrapfig}
\usepackage{url}
\usepackage{tikz}
\usepackage{amsmath}
\usepackage[normalem]{ulem}
\usetikzlibrary{angles,quotes}
\contestname{Hrvatsko otvoreno natjecanje u informatici\\3.\ kolo, 14. prosinca 2019.}
\markright{\textbf{\textsf{Opisi algoritama}}}

\usepackage{hyperref}
\hypersetup{
colorlinks=true,
linkcolor=blue,
filecolor=magenta,
urlcolor=cyan,
}

\begin{document}

\section*{Opisi algoritama}
Zadatke, testne primjere i rješenja pripremili: Fabijan Bošnjak, Nikola
Dmitrović, Marin Kišić, Josip Klepec, Daniel Paleka, Ivan Paljak, Tonko
Sabolčec i Paula Vidas. Primjeri implementiranih rješenja su dani u priloženim
izvornim kodovima.

\subsection*{Zadatak: Koeficijent}
\textsf{Pripremio: Nikola Dmitrović}\\
\textsf{Potrebno znanje: naredba učitavanja i ispisivanja}

Nekoliko je različitih načina na koje možemo riješiti ovaj zadatak.
Najjednostavniji način je ispis izraza oblika $(N-1)+1$. Uočite da ispis
razmaka, koji Python obavezno dodaje, nije dozvoljen što nas dovodi do
činjenice da u naredbi ispisa trebamo koristiti svojstva argumenta
\texttt{sep}.

\textit{Programski kod (pisan u \texttt{Python 3}):}

\vspace{-2ex}
\begin{verbatim}
N = int(input())
print(N-1,'+',1,sep = '')
# može i ovako.. print(N-1,1,sep='+')
# ili ovako... print(str(N-1) + '+' + str(1))
\end{verbatim}

Drugačije rješenje moglo je ići u smjeru kreiranja stringa oblika $1+1+\dots+1$
s ukupno $N$ jedinica.

\textit{Programski kod (pisan u \texttt{Python 3}):}

\vspace{-2ex}
\begin{verbatim}
N = int(input())
s = '1'
for i in range(N-1):
    s += '+1'
print(s)
\end{verbatim}

\subsection*{Zadatak: Hajduk}
\textsf{Pripremio: Fabijan Bošnjak}\\
\textsf{Potrebno znanje: naredba ponavljanja, naredba odlučivanja}

Odredimo najprije koliko je glasova dobio trener s oznakom $1$, a koliko su
glasova dobili ostali treneri. Ovo bismo mogli napraviti tako da, prilikom
učitavanja, varijablu $A$ povećamo za $1$ ako smo učitali broj $1$, a
varijablu $B$ povećavamo za $1$ ako smo učitali broj različit od $1$.

Sve što nam preostaje je odrediti je li trener s oznakom $1$ osvojio barem
polovicu glasova. Ovaj uvjet elegantno možemo provjeriti uspoređivanjem
varijabli $A$ i $B$. Naime, ako je $A \ge B$, tada je trener s oznakom $1$
sigurno osvojio barem polovicu glasova.

Naravno, isti uvjet mogli smo provjeriti uspoređivanjem varijable $A$ s
varijablom $N$ (ukupnim brojem glasova), međutim, naivna implementacija
ovog smjera razmišljanja mogla nas je koštati bodova. Matematički gledano,
uvjet "trener $1$ osvojio je barem polovicu glasova" odgovara izrazu
$A \ge \frac{N}{2}$ kojeg bismo naivno u naš program mogli ukomponirati
u obliku $A >= N / 2$. Greška u dijelu podržanih programskih jezika se krije
u činjenici da znak \texttt{'/'} označava tzv. cjelobrojno dijeljenje. Probajte
sami pronaći primjer u kojem ovakva implementacija nije ispravna.

Problem cjelobrojnog dijeljenja rješavamo jednostavnom algebarskom manipulacijom
pa umjesto $A >= N / 2$ pišemo $2 * A >= N$. Na taj smo način izbjegli
cjelobrojno dijeljenje.

Na kraju još valja upozoriti na rješenja koja se oslanjaju na "pravo dijeljenje",
odnosno, na rad s decimalnim brojevima. U ovom zadatku to uglavnom neće biti
problem, no načelno vam savjetujemo da izbjegavate rad s decimalnim brojevima
ako je to moguće. Više o ovoj temi možete pročitati
\href{http://wiki.xfer.hr/art_epsilon/}{ovdje}.

\subsection*{Zadatak: Preokret}
\textsf{Pripremio: Nikola Dmitrović}\\
\textsf{Potrebno znanje: naredba ponavljanja, rad s nizovima}

Zadatak se sastoji od tri dijela. Za svakog natjecatelja po nešto.  Krenimo
redom. Da bi odrediti koliko je koji tim postigao golova dovoljno je učitavati
zadane vrijednosti i brojati koliko se puta pojavio broj $1$, a koliko broj $2$.

\textit{Programski kod (pisan u \texttt{Python 3}):}

\vspace{-2ex}
\begin{verbatim}
N = int(input())
city = protivnik = 0
for i in range(N):
    gol = int(input())
    if gol == 1:
      city += 1
    else:
      protivnik += 1
print(city, protivnik)
\end{verbatim}

Nastavimo dalje. Da bi odredili i koliko se puta dogodio neriješen rezultat
trebamo pratiti kada je broj postignutih golova jednak, tj. kada će razlika
postignutih golova biti jednaka nuli.

\textit{Programski kod (pisan u \texttt{Python 3}):}

\vspace{-2ex}
\begin{verbatim}
N = int(input())
city = protivnik = 0
nerijeseno = 1 # zbog 0:0
for i in range(N):
    gol = int(input())
    if gol == 1:
      city += 1
    else:
      protivnik += 1
    nerijeseno += (city == protivnik)
print(city, protivnik)
print(nerijeseno)
\end{verbatim}

Za treći dio zadatka uočimo da se preokret sastoji od tri dijela. Prvo, jedan
od timova gubi i počne postizati golove. Drugo, nakon niza postignutih golova
dođe do neriješenog rezultata. Treće, tu ne stane već nastavi davati golove.
Znači, razlika postignutih golova prvo pada do nule i onda nastavi rasti. Ili
obrnuto, ovisno jel li City pravi preokret ili njegov protivnik. Jednu od
implementacija rješenja možete pronaći u priloženim izvornim kodovima.

\clearpage

\subsection*{Zadatak: Grudanje}
\textsf{Pripremio: Marin Kišić}\\
\textsf{Potrebno znanje: prefiks sume, binarna pretraga}

Za $14$ bodova bilo je potrebano samo simulirati ono što piše u zadatku. Jednom
for-petljom ćemo slovo po slovo označavati s \texttt{‘*’}. Nakon toga ćemo
drugom for-petljom proći po svim intervalima te za svaki interval trećom
for-petljom provjeriti je li savršen.

Vremenska složenost ovog dijela je $\mathcal{O}(NQN)=\mathcal{O}(QN^2)$.

Za dodatnih $14$ bodova bilo je potrebno poznavati ideju prefiks suma. Recimo
da za svako slovo imamo niz $pref$ u kojem će na poziciji $i$ pisati broj tog
slova od početka riječi do $i$-tog slova. Primjerice, za riječ \texttt{abcaab}
i slovo \texttt{'a'}, niz $pref$ bio bi $[1, 1, 1, 2, 3, 3]$. Niz $pref$ nam je
koristan jer sada možemo u $\mathcal{O}(1)$ odrediti kolko puta se to slovo
pojavljuje u intervalu od
$L$-tog slova do $R$-tog slova u riječi. Formula je $pref[R]-pref[L-1]$.

Sad je ideja da prije prolaska po intervalima izračunamo niz pref za svako
slovo. Zatim, kada prolazimo po intervalima ćemo, uz pomoć niza $pref$, za svako
slovo odrediti kolko ga ima u trenutnom intervalu.

Vremenska složenost ovog dijela je $\mathcal{O}(N(N\Sigma * Q\Sigma))=
\mathcal{O}((N^2 + Q)\Sigma)$, gdje $\Sigma$ označava veličinu abecede
(u našem slučaju $\Sigma=26$).

Za sve bodove bilo je potrebno primijetiti sljedeće: ako je riječ savršena nakon
$i$-te grude, tada će onda biti savršena i nakon svake grude poslije $i$-te
grude. To svojstvo nam omogućava da binarnom pretragom pronađemo prvu grudu
nakon koje riječ postane savršena. Provjeru u binarnoj pretrazi radit ćemo kao
i u prethodnoj parcijali (uz pomoć prefiks suma).

Vremenska složenost: $\mathcal{O}((N\Sigma + Q\Sigma)\log N) = \mathcal{O}((N \log N + Q)\Sigma)$.

\subsection*{Zadatak: Drvca}
\textsf{Pripremili: Marin Kišić i Josip Klepec}\\
\textsf{Potrebno znanje: ad-hoc, ugrađene strukture podataka}

Formalnim rječnikom, zadatak je bio razdvojiti zadanih $N$ brojeva u $2$
aritmetička niza.

Za 20 bodova bilo je dovoljno u $\mathcal{O}(2^N)$ fiksirati neku bitmasku u
kojoj vrijednost $i$-tog bita govori u kojem se redu nalazi $i$-ti broj. Nakon
tog trebalo je još provjeriti jesu li redovi aritmetički nizovi.

Za dodatnih $30$ bodova bilo je potrebno primijetiti da će najmanji broj iz
niza biti i najmanji broj u jednom od redova. Sada, najmanji broj stavimo kao
prvi broj u prvi red, fiksiramo neki broj iz niza koji ćemo staviti na drugo
mjesto u prvom redu. Sada, kada znamo prvi i drugi broj prvog reda, znamo i
razliku između svaka dva susjedna broja u prvom redu pa možemo dodavati broj po
broj u prvi red tako dugo dok broj koji želimo dodati postoji u originalnom
nizu te  nakon svakog dodavanja provjerimo tvore li svi brojevi koje nismo
stavili u prvi red aritmetički niz.  Ako je tako, onda smo našli rješenje. Ako
nakon isprobavanja svih mogućih brojeva za drugi broj u prvom redu nismo našli
rješenje, onda ono ni ne postoji.

Vremenska složenost je $\mathcal{O}(N^3)$.

U stilu Boba Graditelja: \emph{“Mozemo li to popraviti? Naravno da mozemo!”}

Potrebno je primijetiti da za prva dva broja u prvom redu ne trebamo isprobavati
sve kombinacije (najmanji broj, neki broj iz niza), već samo tri kombinacije.
Promotrimo tri najmanja broja iz niza, označimo ih s $A \le B \le C$. Sigurno
znamo da, ako postoji rješenje, jedan od redova će početi s $(A, B)$ ili $(A, C)$
ili $(B, C)$.

Vremenska složenost je $\mathcal{O}(N^2)$.

Za treći podzadatak možemo probati započeti jedan od redova na sva tri gore
opisana načina, napraviti niz od točno $\frac{N}{2}$ brojeva i onda provjeriti
tvori li ostalih $\frac{N}{2}$ brojeva aritmetički niz.

Vremenska složenost je $\mathcal{O}(N)$.

Za sve bodove fiksirat ćemo na gore spomenuta tri načina početak jednog reda.
Zatim ćemo dodavati broj po broj u taj red te provjeriti tvore li svi brojevi
koji su ostali aritmetički niz. Tu provjeru moramo napraviti pametnije od
naivnog rješenja for petljom. Održavat ćemo dva skupa. U jednom ćemo pamtiti sve
brojeve koji su ostali, a u drugom razlike između susjednih brojeva koji su u
prvom skupu. Kada broj izbacimo iz prvog skupa, tj. dodamo ga u prvi red, onda
moramo iz drugog skupa izbaciti razlike između njega i brojeva susjednih njemu,
ali i ubaciti razliku između brojeva koji su mu bili susjedi jer su sada oni
posatli susjedi. Kada imamo ta dva skupa i znamo kako ih održavati, možemo samo
u skupu razlika provjeriti je li najveći broj jednak najmanjem. Ako jest, onda
su svi brojevi u skupu jednaki, a to znači da je razlika između svaka dva
susjedna broja koja su ostala jednaka, a to znači da ti brojevi tvore
aritmetički niz, odnosno da smo pronašli rješenje. Sve operacije nad
spomenutim skupovima može podržati kolekcija \texttt{std::set} u jeziku
C++. Slične kolekcije postoje i u ostalim podržanim jezicima uz iznimku
programskog jezika C.

Vremenska složenost je $\mathcal{O}(N \log N)$.

\subsection*{Zadatak: Lampice}
\textsf{Pripremio: Tonko Sabolčec}\\
\textsf{Potrebno znanje: hashiranje, binarno pretraživanje, centroidna
dekompozicija stabla}

Prvi podzadatak moguće je riješiti na više načina. Opisat ćemo rješenje pomoću
hashiranja koje će nam koristiti i za konačno rješenje. Ukorijenimo stablo u
čvoru $R$. Zanimaju nas svi palindromski segmenti kojima je jedan kraj u čvoru
$R$.
Za svaki čvor $X$ računamo dvije hash vrijednosti:

\[down_x = x_0B^{k-1} + x_1B^{k-2} + \dots + x_{k-1}B^0\]
\[up_x = x_0B^0 + x_1B^1 + \dots + x_{k-1}B^{k-1}\]

Pri tome je $x_0$, $x_1$, \dots, $x_{k-1}$ niz boja na putu od korijena $R$ do
čvora $X$ ($color(R) = x_0$, $color(X) = x_{k-1}$), a $B$ vrijednost baze
hashiranja. Ove dvije vrijednosti mogu se jednostavno odrediti jednim DFS
obilaskom po stablu. Put od žaruljice $R$ do žaruljice $X$ je palindromski
segment ako vrijedi $down_X = up_X$.  Ako ovaj postupak ponovimo za svaki
mogući korijen stabla, pokrili smo sve slučajeve. Složenost tog algoritma
iznosi $\mathcal{O}(N^2)$.

Drugi podzadatak klasični je problem traženja najduljeg palindroma u nizu, koji
se moze riješiti pomoću
\href{https://en.wikipedia.org/wiki/Longest_palindromic_substring}{Manacherovog
algoritma}.

Treći podzadatak zapravo je jednak drugom podzadatku samo što Manacherov algoritam
moramo primijeniti za svaki par listova u stablu, kojih zbog ograničenja ima
dovoljno malo.

Potpuno rješenje započet ćemo sljedećom zamjedbom: Ako postoji palindromski
segment duljine $K > 2$, tada postoji i palindromski segment duljine $K - 2$.
Drugim riječima, traženu duljinu moguće je pronaći binarnim pretraživanjem za
segmente parne i neparne duljine.

No, kako provjeriti postoji li palindrom određene duljine u nekom stablu? Kako
bismo odgovorili na to pitanje, prvo ćemo riješiti lakšu verziju tog problema --
provjerit ćemo postoji li palindrom zadane duljine koji prolazi korijenom
(unaprijed određenim čvorom) stabla, čvorom $R$.

Na ukorijenjenom stablu, put između neka dva čvora $A$ i $B$ sastoji se od
dvije grane na stablu, pri čemu je duljina jedne grane veća ili jednaka duljini
druge grane. Pretpostavimo da se čvor $A$ nalazi na duljoj grani. Put između
čvorova $A$ i $B$ rastavit cemo na $3$ dijela: put od $A$ do $C$, put od $C$ do
korijena stabla i put od korijena stabla do čvora $B$, pri čemu je $C$ odabran
tako da su duljine $A-C$ i $R-B$ jednake. Primijetite kako je $C$ jedinstveno
određen čvorom $A$, budući da nas zanimaju samo palindromi fiksne duljine. Kako
bismo provjerili je li put od $A$ do $B$ palindromski segment, dovoljno je
provjeriti sljedeće:

\begin{itemize}
  \item (1) Put od $R$ do $C$ je palindrom (na slici označeno crveno)
  \item (2) Slijed boja na putu od $C$ do $A$ jednak je slijedu boja na putu od $R$
    do $B$ (na slici označeno plavom).
\end{itemize}

\begin{figure}[!htbp]
\centering
\includegraphics[width=0.4\textwidth]{pic/opisi_lampice.png}
\end{figure}

Za svaki čvor stabla unaprijed odredimo hash vrijednosti $down_X$ i $up_X$ na
isti način kao što je opisano u rješenju prvog podzadatka. Prvu provjeru
jednostavno radimo uspoređivanjem vrijednosti $down_C$ i $up_C$. Drugu provjeru
moguće je napraviti usporedbom vrijednosti:

\begin{itemize}
  \item $down_B$
  \item (3) $down_A - down_{par(C)} \cdot B^{dep(A)-dep(C)}$, gdje je $par(C)$ roditelj
        čvora $C$, a $dep(X)$ dubina čvora $X$.
\end{itemize}

Sada znamo efikasno provjeriti je li put od čvora $A$ do čvora $B$ palindromski
segment, ali to i dalje nije dovoljno brzo jer postoji $\mathcal{O}(N^2)$
mogućih parova $(A, B)$. Pokušajmo promatrati problem iz malo drugačijeg kuta:
Ako nam je poznat čvor $A$, postoji li (barem jedan) čvor $B$ koji zadovoljava
navedena svojstva?  Neka je $S_B$ skup hasheva u koji ćemo staviti vrijednosti
$down_B$ svih čvorova $B$ u promatranom stablu. Tada za neki čvor $A$ možemo
provjeriti postoji li odgovarajući čvor $B$ tako da provjerimo uvjet (1) i
postoji li hash vrijednost (3) u skupu $S_B$. Pretpostavlja se da je složenost
operacija dodavanja vrijednosti u skup i provjeru postoji li neka vrijednost u
skupu $\mathcal{O}(1)$ koristenjem hash tablice (\texttt{std::unordered\_set} u
jeziku C++). Složenost cijele provjere iznosi $\mathcal{O}(n)$.

\textbf{Napomena:} prilikom implementacije potrebno je paziti da najviši
zajednički predak čvorova iz skupa $S_B$ i čvora $A$ bude upravo korijen $R$.

Sada kada znamo postoji li palindromski segment određene duljine koji prolazi
korijenom stabla $R$, možemo primijeniti centroidnu dekompoziciju. Ako za svako
dekomponirano stablo napravimo provjeru s centroidom kao korijenom stabla,
pokrit ćemo sve slučajeve, odnosno moći ćemo provjeriti postoji li palindrom
određene duljine u cijelom stablu. Složenost takve provjere iznosi $\mathcal{O}(n \log n)$.
Dodavanje razine binarnog pretraživanja dovodi do konačnog rješenja složenosti
$\mathcal{O}(n \log^2 n)$.

\clearpage

\subsection*{Zadatak: Sob}
\textsf{Pripremili: Paula Vidas i Daniel Paleka}\\
\textsf{Potrebno znanje: matematika, pohlepni algoritmi}

Uređeni par $(a, b)$ zvat ćemo \emph{dobrim} ako vrijedi $a \mathbin\& b = a$.

Prvi podzadatak možemo riješiti tako da $a \in A$ uparimo sa onim $b \in B$ za
kojeg vrijedi $b \mathbin{\textrm{mod}} N = a$.

Drugi podzadatak možemo riješiti sljedećim algoritmom: Neka su
$i_1 > i_2 > ... > i_k$ pozicije jedinica u binarnom zapisu od $N$. Uparit ćemo
najmanjih $2^{i_1}$ elemenata skupova $A$ i $B$ tako da uparimo one $a$ i $b$
za koje vrijedi $a \equiv b \mod 2^{i_1}$. Zatim uzmemo sljedećih $2^{i_2}$
najmanjih elemenata i uparimo one koji su jednaki modulo $2^{i_2}$, itd.
Dokaz da su odabrani parovi dobri ostavljamo čitatelju za vježbu.

Treći podzadatak mogao se riješiti na više načina. Jedan mogući način je da
napravimo bipartitni graf sa čvorovima iz skupova $A$ i $B$ te dodamo bridove
između svih dobrih parova. Na dobivenom grafu napravimo algoritam za uparivanje
na bipartitnom grafu (\emph{bipartite matching}), na primjer u složenosti
$\mathcal{O}(NE)$, pri čemu je $E$ broj bridova u grafu.
Primjetimo da broj bridova možemo ograničiti sa $E < 3^{10} = 59049$.

Drugi način koristi sljedeći pohlepni algoritam: Prolazimo kroz elemente skupa
$A$ od većih prema manjima i trenutni element uparimo sa najmanjim još
neuparenim elementom skupa $B$ s kojim ga smijemo upariti.

Ako pokrenemo taj algoritam na nekoliko primjera, možemo uočiti sljedeću
pravilnost: Neka se najveći element skupa $A$, tj. $N - 1$, upari sa $b \in B$.
Tada se upare i $N - 1 - t$ sa $b - t$ za svaki $t \in \{1, 2, ..., b - M\}$.
Nakon što maknemo uparene elemente dobili smo isti zadatak, sada za skupove
$A' = \{0, 1, ..., N - 1 - (b - M) - 1\}$ i
$B' = \{b + 1, b + 2, ..., M + N - 1\}$. Ovo rješenje možemo implementirati u
složenosti $\mathcal{O}(N)$.

Dokaz prethodne tvrdnje:\\
Neka je $a = N - 1$ (radi ljepših oznaka) i $b$, kao i prije, najmanji element
skupa $B$ za kojeg vrijedi $a \mathbin\& b = a$. Indeksom $i$ ćemo označavati znamenku
težine $2^i$. Ako je $b = M$ nemamo što za dokazivati, pa pretpostavimo da je
$b > M$ i označimo $k = b - M$. Neka je $i$ pozicija najmanje značajne jedinice u $b$.
Očito mora biti $a_j = b_j = 0$ za $j < i$. Kada bi bilo $a_i = 0$ onda bi vrijedilo
$a \mathbin\& (b - 1) = a$ pa $b$ ne bi bio najmanji element od $B$ koji se može upariti
sa $a$. Dakle, $a_i = b_i = 1$. Sada je očito da je $(a - t, b - t)$ dobar par za
$t \in \{1, 2, ..., 2^i\}$. Ako je $k \leq 2^i$ gotovi smo, inače promatramo sljedeću
najmanje značajnu jedinicu u $b$ i induktivno ponavljamo isti postupak.
Preostaje još pokazati da uvijek postoji neki $b \in B$ s kojim se $a$ može upariti, no
to ostavljamo čitateljici za vježbu.


\end{document}
%%% Local Variables:
%%% mode: latex
%%% mode: flyspell
%%% ispell-local-dictionary: "croatian"
%%% End:
