\documentclass[a4paper]{article}
\usepackage{zadaci}
\usepackage{wrapfig}
\usepackage{url}
\usepackage{tikz}
\usepackage{amsmath}
\usepackage[normalem]{ulem}
\usetikzlibrary{angles,quotes}
\contestname{Hrvatsko otvoreno natjecanje u informatici\\3.\ kolo, 14. studenog 2019.}
\markright{\textbf{\textsf{Opisi algoritama}}}

\begin{document}

\section*{Opisi algoritama}
Zadatke, testne primjere i rješenja pripremili: 
Primjeri implementiranih rješenja su dani u priloženim izvornim kodovima.

\subsection*{Zadatak: Sob}
\textsf{Pripremili: Paula Vidas i Daniel Paleka}\\
\textsf{Potrebno znanje: matematika, pohlepni algoritmi}

Uređeni par $(a, b)$ zvat ćemo \emph{dobrim} ako vrijedi $a \mathbin\& b = a$.

Prvi podzadatak možemo riješiti tako da $a \in A$ uparimo sa onim $b \in B$ za
kojeg vrijedi $b \mathbin{\textrm{mod}} N = a$.

Drugi podzadatak možemo riješiti sljedećim algoritmom: Neka su
$i_1 > i_2 > ... > i_k$ pozicije jedinica u binarnom zapisu od $N$. Uparit ćemo
najmanjih $2^{i_1}$ elemenata skupova $A$ i $B$ tako da uparimo one $a$ i $b$
za koje vrijedi $a \equiv b \mod 2^{i_1}$. Zatim uzmemo sljedećih $2^{i_2}$
najmanjih elemenata i uparimo one koji su jednaki modulo $2^{i_2}$, itd.
Dokaz da su odabrani parovi dobri ostavljamo čitatelju za vježbu.

Treći podzadatak mogao se riješiti na više načina. Jedan mogući način je da
napravimo bipartitni graf sa čvorovima iz skupova $A$ i $B$ te dodamo bridove
između svih dobrih parova. Na dobivenom grafu napravimo algoritam za uparivanje
na bipartitnom grafu (\emph{bipartite matching}), na primjer u složenosti
$\mathcal{O}(NE)$, pri čemu je $E$ broj bridova u grafu.
Primjetimo da broj bridova možemo ograničiti sa $E < 3^{10} = 59049$.

Drugi način koristi sljedeći pohlepni algoritam: Prolazimo kroz elemente skupa
$A$ od većih prema manjima i trenutni element uparimo sa najmanjim još
neuparenim elementom skupa $B$ s kojim ga smijemo upariti.

Ako pokrenemo taj algoritam na nekoliko primjera, možemo uočiti sljedeću
pravilnost: Neka se najveći element skupa $A$, tj. $N - 1$, upari sa $b \in B$.
Tada se upare i $N - 1 - t$ sa $b - t$ za svaki $t \in \{1, 2, ..., b - M\}$.
Nakon što maknemo uparene elemente dobili smo isti zadatak, sada za skupove
$A' = \{0, 1, ..., N - 1 - (b - M) - 1\}$ i
$B' = \{b + 1, b + 2, ..., M + N - 1\}$. Ovo rješenje možemo implementirati u
složenosti $\mathcal{O}(N)$.

Dokaz prethodne tvrdnje:\\
Neka je $a = N - 1$ (radi ljepših oznaka) i $b$, kao i prije, najmanji element
skupa $B$ za kojeg vrijedi $a \mathbin\& b = a$. Indeksom $i$ ćemo označavati znamenku
težine $2^i$. Ako je $b = M$ nemamo što za dokazivati, pa pretpostavimo da je
$b > M$ i označimo $k = b - M$. Neka je $i$ pozicija najmanje značajne jedinice u $b$.
Očito mora biti $a_j = b_j = 0$ za $j < i$. Kada bi bilo $a_i = 0$ onda bi vrijedilo
$a \mathbin\& (b - 1) = a$ pa $b$ ne bi bio najmanji element od $B$ koji se može upariti
sa $a$. Dakle, $a_i = b_i = 1$. Sada je očito da je $(a - t, b - t)$ dobar par za
$t \in \{1, 2, ..., 2^i\}$. Ako je $k \leq 2^i$ gotovi smo, inače promatramo sljedeću
najmanje značajnu jedinicu u $b$ i induktivno ponavljamo isti postupak.
Preostaje još pokazati da uvijek postoji neki $b \in B$ s kojim se $a$ može upariti, no
to ostavljamo čitateljici za vježbu.


\end{document}
%%% Local Variables:
%%% mode: latex
%%% mode: flyspell
%%% ispell-local-dictionary: "croatian"
%%% End:
