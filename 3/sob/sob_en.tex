%%%%%%%%%%%%%%%%%%%%%%%%%%%%%%%%%%%%%%%%%%%%%%%%%%%%%%%%%%%%%%%%%%%%%%
% Problem statement
\begin{statement}[
  problempoints=110,
  timelimit=1 second,
  memorylimit=512 MiB,
]{Sob}

\setlength\intextsep{-0.1cm}
\begin{wrapfigure}[9]{r}{0.17\textwidth}
\centering
\includegraphics[width=0.17\textwidth]{img/sob.png}
\end{wrapfigure}

It was a dark and dreary Christmas Eve when our hero pondered, weak and weary,
over a quaint and curious COCI task. When he nodded, nearly napping, suddenly
he heard a tapping, tapping and a mighty roar. A giant reindeer broke through
his chamber door, merely this and nothing more. While our hero's heart slightly
  fluttered, the beast simply uttered: \textit{``I won't leave until you solve this problem''}.

In the problem you were given two integers $N$ and $M$ and you were supposed
to perfectly match the numbers from sets $A = \{ 0, 1, 2, \dots, N - 1 \}$ and
$B = \{ M, \dots, M + N - 1\}$ into $N$ pairs, such that for the matched numbers
$x \in A$ and $y \in B$ it holds $x \mathbin{\&} y = x$, where $\&$ denotes a
bitwise AND operation.

%%%%%%%%%%%%%%%%%%%%%%%%%%%%%%%%%%%%%%%%%%%%%%%%%%%%%%%%%%%%%%%%%%%%%%
% Input
\subsection*{Input}
The first line contains two integers $N$ and $M$
$(1 \le N \le M, N + M \le 10^6)$ from the task description.

%%%%%%%%%%%%%%%%%%%%%%%%%%%%%%%%%%%%%%%%%%%%%%%%%%%%%%%%%%%%%%%%%%%%%%
% Output
\subsection*{Output}
You should output $N$ lines and in each line you should output two integers $x$
and $y$, where $x$ belongs to set $A$ and $y$ belongs to set $B$. Numbers in
each line should correspond to one of the matched pairs from task description.

It is possible to prove that the solution always exists.

%%%%%%%%%%%%%%%%%%%%%%%%%%%%%%%%%%%%%%%%%%%%%%%%%%%%%%%%%%%%%%%%%%%%%%
% Scoring
 \subsection*{Scoring}
{\renewcommand{\arraystretch}{1.4}
  \setlength{\tabcolsep}{6pt}
  \begin{tabular}{ccl}
 Subtask & Score & Constraints \\ \midrule
  1 & 10 & $N$ is a power of $2$ \\
  2 & 29 & $N + M$ is a power of $2$ \\
  3 & 39 & $N + M \le 1000$ \\
  4 & 32 & No additional constraints.
\end{tabular}}

%%%%%%%%%%%%%%%%%%%%%%%%%%%%%%%%%%%%%%%%%%%%%%%%%%%%%%%%%%%%%%%%%%%%%%
% Examples
\subsection*{Examples}
\begin{tabularx}{\textwidth}{X'X'X}
\sampleinputs{test/sob.dummy.in.1}{test/sob.dummy.out.1} &
\sampleinputs{test/sob.dummy.in.2}{test/sob.dummy.out.2} &
\sampleinputs{test/sob.dummy.in.3}{test/sob.dummy.out.3}
\end{tabularx}

%%%%%%%%%%%%%%%%%%%%%%%%%%%%%%%%%%%%%%%%%%%%%%%%%%%%%%%%%%%%%%%%%%%%%%
% We're done
\end{statement}

%%% Local Variables:
%%% mode: latex
%%% mode: flyspell
%%% ispell-local-dictionary: "croatian"
%%% TeX-master: "../hio.tex"
%%% End:
