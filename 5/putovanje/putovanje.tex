%%%%%%%%%%%%%%%%%%%%%%%%%%%%%%%%%%%%%%%%%%%%%%%%%%%%%%%%%%%%%%%%%%%%%%
% Problem statement
\begin{statement}[
  problempoints=110,
  timelimit=1 sekunda,
  memorylimit=512 MiB,
]{Putovanje}

Mali Fabijan jako voli putovati. On želi popiti \sout{piv} kavu u svakom od $N$
gradova označenih brojevima od $1$ do $N$. Gradovi su međusobno povezani s $(N
- 1)$ dvosmjernih cesta na način da se iz svakog grada može doći u svaki drugi
putujući tim cestama. On će kave ispijati redom u gradovima od $1$ do $N$ (svoje
putovanje započinje u gradu $1$, a završava u gradu $N$). Najprije iz grada
$1$ (u kojem je već popio kavu) putuje u grad $2$ kako bi mogao tamo popiti
kavu. Pritom će možda morati proći i kroz neke druge gradove (kroz koje je
možda i ranije prošao), ali u njima neće stati kako bi popio kavu.  Nakon što
je popio kavu u gradu $2$ putuje do grada $3$ i tako dalje dok ne popije
posljednju kavu u gradu $N$.

Da bi prešao neku cestu, Fabijan mora imati kratu za nju. Cesta $i$ na
raspolaganju ima dvije karte: jednokratnu koja košta $C_{i1}$ kuna i
višekratnu koja košta $C_{i2}$ kuna. Za svaku cestu on može ili svaki puta
kada ju prelazi kupiti jednokratnu kartu ili jednom kupiti višekratnu kartu
te više nikada ne kupovati kartu za nju.

Budući da Fabijan nije dobio stipendiju, ne zna hoće li si moći priuštiti ovo
putovanje. Odredite kolika je najmanja cijena njegovog putovanja ako karte
kupuje optimalno.


%%%%%%%%%%%%%%%%%%%%%%%%%%%%%%%%%%%%%%%%%%%%%%%%%%%%%%%%%%%%%%%%%%%%%%
% Input
\subsection*{Ulazni podaci}
U prvom se retku nalazi prirodan broj $N$ $(2 \le N \le 200\ 000)$ iz teksta
zadatka.

U $i$-tom od sljedećih $(N - 1)$ redaka nalaze se po četiri prirodna broja
$A_i$, $B_i$, $C_{i1}$, $C_{i2}$ $(1 \le A_i, B_i \le N, 1 \le C_{i1} \le
C_{i2} \le 100\ 000)$ koji označavaju da su gradovi $A_i$ i $B_i$ povezani
cestom čije cijene karata su   $C_{i1}$ i $C_{i2}$.

%%%%%%%%%%%%%%%%%%%%%%%%%%%%%%%%%%%%%%%%%%%%%%%%%%%%%%%%%%%%%%%%%%%%%%
% Output
\subsection*{Izlazni podaci}
U jedini redak ispišite najmanju cijenu putovanja.

%%%%%%%%%%%%%%%%%%%%%%%%%%%%%%%%%%%%%%%%%%%%%%%%%%%%%%%%%%%%%%%%%%%%%%
% Scoring
 \subsection*{Bodovanje}
{\renewcommand{\arraystretch}{1.4}
  \setlength{\tabcolsep}{6pt}
  \begin{tabular}{ccl}
 Podzadatak & Broj bodova & Ograničenja \\ \midrule
  1 & 20 & $2 \le N \le 2000$ \\
  2 & 25 & Svaki će grad biti povezan s najviše dva druga grada. \\
  3 & 65 & Nema dodatnih ograničenja.
\end{tabular}}

%%%%%%%%%%%%%%%%%%%%%%%%%%%%%%%%%%%%%%%%%%%%%%%%%%%%%%%%%%%%%%%%%%%%%%
% Examples
\subsection*{Probni primjeri}
\begin{tabularx}{\textwidth}{X'X'X}
\sampleinputs{test/putovanje.dummy.in.1}{test/putovanje.dummy.out.1} &
\sampleinputs{test/putovanje.dummy.in.2}{test/putovanje.dummy.out.2} &
\sampleinputs{test/putovanje.dummy.in.3}{test/putovanje.dummy.out.3}
\end{tabularx}

\clearpage

\textbf{Pojašnjenje prvog probnog primjera:}

Fabijan prvo putuje od grada $1$ do grada $2$ te mu je optimalno kupiti
višekratnu kartu ($5$ kuna) za cestu koja ih povezuje. Zatim putuje od grada
$2$ preko grada $1$ do grada $3$. Ima višekratnu kartu za cestu od grada $2$ do
grada $1$, a za cestu od grada $1$ do grada $3$ može kupiti jednokratnu kartu
($2$ kune). Na putu od grada $3$ do grada $4$ kupuje još jednu jednokratnu
kartu za cestu koja povezuje gradove $3$ i $2$ ($2$ kune) te kupuje jednokratnu
kartu za cestu koja povezuje gradove $2$ i $4$ ($1$ kuna). Sve zajedno je
potrošio $10$ kuna.

%%%%%%%%%%%%%%%%%%%%%%%%%%%%%%%%%%%%%%%%%%%%%%%%%%%%%%%%%%%%%%%%%%%%%%
% We're done
\end{statement}

%%% Local Variables:
%%% mode: latex
%%% mode: flyspell
%%% ispell-local-dictionary: "croatian"
%%% TeX-master: "../hio.tex"
%%% End:
