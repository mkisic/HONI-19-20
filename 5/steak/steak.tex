%%%%%%%%%%%%%%%%%%%%%%%%%%%%%%%%%%%%%%%%%%%%%%%%%%%%%%%%%%%%%%%%%%%%%%
% Problem statement
\begin{statement}[
  problempoints=20,
  timelimit=1 sekunda,
  memorylimit=512 MiB,
]{Steak}

%\setlength\intextsep{-0.1cm}
%\begin{wrapfigure}[3]{r}{0.26\textwidth}
%\centering
%\includegraphics[width=0.20\textwidth]{img/steak.png}
%\end{wrapfigure}

Ivan voli jesti odreske. Jednom je u intervjuu za \textit{Gloriju} izjavio da svake
godine točno $N$ puta pojede po jedan srednje pečeni primjerak. Što se tiče
priloga, tu je malo neodlučan. Ne može odlučiti je li bolje jesti mrkvu ili
brokulu. Zato je odlučio da će, ako odrezak jede na parni dan u godini,
prilog biti brokula, a ako ga jede na neparni dan, prilog biti mrkva.

\textbf{Dogovor} -- svaki mjesec u godini ima $30$ dana.

\textbf{Definicija} -- dan je paran (odnosno neparan) ako je njegov redni broj
u godini paran (odnosno neparan). Primjerice, 8. veljače je paran jer je 38.
dan u godini, dok je 25.  rujna neparan jer je 265. dan u godini.

Za svaki od $N$ obroka znamo dan $D$ i mjesec $M$ kada je Ivan objedovao.
Napišite program koji će za svaki zadani dan ispisati što je bio prilog uz
odrezak.

%%%%%%%%%%%%%%%%%%%%%%%%%%%%%%%%%%%%%%%%%%%%%%%%%%%%%%%%%%%%%%%%%%%%%%
% Input
\subsection*{Ulazni podaci}
U prvom je retku prirodan broj $N$ $(1 \le N \le 100)$ iz teksta zadatka.

U sljedećih $2N$ redaka su parovi podataka koji opisuju $i$-ti dan kada je Ivan
jeo odrezak. Pri tome vrijedi da je u prvom retku (para) prirodan broj $D$
$(1 \le D \le 30)$, a u drugom retku riječ $M$ (\texttt{SIJECANJ},
\texttt{VELJACA}, \texttt{OZUJAK}, \texttt{TRAVANJ}, \texttt{SVIBANJ},
\texttt{LIPANJ}, \texttt{SRPANJ}, \texttt{KOLOVOZ}, \texttt{RUJAN},
\texttt{LISTOPAD}, \texttt{STUDENI} ili \texttt{PROSINAC}), redom dan i naziv
mjeseca u godini kada se jeo odrezak.

%%%%%%%%%%%%%%%%%%%%%%%%%%%%%%%%%%%%%%%%%%%%%%%%%%%%%%%%%%%%%%%%%%%%%%
% Output
\subsection*{Izlazni podaci}
U $i$-ti od $N$ redaka ispišite naziv priloga koji se jeo uz odrezak tijekom
$i$-tog jedenja odreska. Nazivi priloga su \texttt{"BROKULA"} ili
\texttt{"MRKVA"} (bez navodnika, velika slova).


%%%%%%%%%%%%%%%%%%%%%%%%%%%%%%%%%%%%%%%%%%%%%%%%%%%%%%%%%%%%%%%%%%%%%%
% Scoring
\subsection*{Bodovanje}
U testnim primjerima vrijednima $10$ bodova, vrijedit će $N=3$.

%%%%%%%%%%%%%%%%%%%%%%%%%%%%%%%%%%%%%%%%%%%%%%%%%%%%%%%%%%%%%%%%%%%%%%
% Examples
\subsection*{Probni primjeri}
\begin{tabularx}{\textwidth}{X'X'X}
\sampleinputs{test/steak.dummy.in.1}{test/steak.dummy.out.1} &
\sampleinputs{test/steak.dummy.in.2}{test/steak.dummy.out.2} &
\sampleinputs{test/steak.dummy.in.3}{test/steak.dummy.out.3}
\end{tabularx}

  %\textbf{Pojašnjenje prvog probnog primjera:}

%21. siječnja je 21. dan u godini. Radi se o neparnom danu pa Ivan jede mrkvu.\\
%19. veljače je 49. dan u godini. Radi se o neparnom danu pa Ivan jede mrkvu.\\
%16. lipnja je 166. dan u godini. Radi se o parnom danu pa Ivan jede brokulu.\\

%%%%%%%%%%%%%%%%%%%%%%%%%%%%%%%%%%%%%%%%%%%%%%%%%%%%%%%%%%%%%%%%%%%%%%
% We're done
\end{statement}

%%% Local Variables:
%%% mode: latex
%%% mode: flyspell
%%% ispell-local-dictionary: "croatian"
%%% TeX-master: "../hio.tex"
%%% End:
