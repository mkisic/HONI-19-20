%%%%%%%%%%%%%%%%%%%%%%%%%%%%%%%%%%%%%%%%%%%%%%%%%%%%%%%%%%%%%%%%%%%%%%
% Problem statement
\begin{statement}[
  problempoints=110,
  timelimit=1 sekunda,
  memorylimit=512 MiB,
]{Zapina}

Čak $N$ mladih informatičara priprema se za drugi dio natjecateljske sezone na
zimskom kampu mladih informatičara u \sout{Krapini} Zagrebu. Gospodin Malnar,
veliki zagovornik reda, rada i discipline, poslagao je informatičare u red i
svakome dao nekoliko (možda i nula) zadataka. Ukupno je podijelio \textbf{N
različitih} zadataka te zna da, ako je $i$-tom informatičaru u redu dao točno
$i$ zadataka, tada je taj informatičar sretan.

Na koliko je različitih načina gospodin Malnar mogao raspodijeliti zadatke tako
da je \textbf{barem jedan} informatičar bio sretan? Dva načina su različita
ako postoje informatičar i zadatak takvi da je u jednom načinu informatičar
dobio taj zadatak, a u drugom nije.

%%%%%%%%%%%%%%%%%%%%%%%%%%%%%%%%%%%%%%%%%%%%%%%%%%%%%%%%%%%%%%%%%%%%%%
% Input
\subsection*{Ulazni podaci}
U prvom je retku prirodan broj $N$ $(1 \le N \le 350)$ iz teksta zadatka.

%%%%%%%%%%%%%%%%%%%%%%%%%%%%%%%%%%%%%%%%%%%%%%%%%%%%%%%%%%%%%%%%%%%%%%
% Output
\subsection*{Izlazni podaci}
Ispišite ostatak pri dijeljenju traženog broja načina s $10^9+7$.

%%%%%%%%%%%%%%%%%%%%%%%%%%%%%%%%%%%%%%%%%%%%%%%%%%%%%%%%%%%%%%%%%%%%%%
% Scoring
 \subsection*{Bodovanje}
{\renewcommand{\arraystretch}{1.4}
  \setlength{\tabcolsep}{6pt}
  \begin{tabular}{ccl}
 Podzadatak & Broj bodova & Ograničenja \\ \midrule
  1 & 22 & $1 \le N \le 7$ \\
  2 & 33 & $1 \le N \le 20$ \\
  3 & 55 & Nema dodatnih ograničenja.
\end{tabular}}

%%%%%%%%%%%%%%%%%%%%%%%%%%%%%%%%%%%%%%%%%%%%%%%%%%%%%%%%%%%%%%%%%%%%%%
% Examples
\subsection*{Probni primjeri}
\begin{tabularx}{\textwidth}{X'X'X}
\sampleinputs{test/zapina.dummy.in.1}{test/zapina.dummy.out.1} &
\sampleinputs{test/zapina.dummy.in.2}{test/zapina.dummy.out.2} &
\sampleinputs{test/zapina.dummy.in.3}{test/zapina.dummy.out.3}
\end{tabularx}

\textbf{Pojašnjenje drugog probnog primjera:}

Načini raspodjele u kojima je barem jedan informatičar sretan su:
\begin{enumerate}
  \item Prvi zadatak prvom informatičaru u redu, a drugi drugom.
  \item Drugi zadatak prvom infromatičaru u redu, a prvi drugom.
  \item Oba zadatka drugom informatičaru u redu.
\end{enumerate}

%%%%%%%%%%%%%%%%%%%%%%%%%%%%%%%%%%%%%%%%%%%%%%%%%%%%%%%%%%%%%%%%%%%%%%
% We're done
\end{statement}

%%% Local Variables:
%%% mode: latex
%%% mode: flyspell
%%% ispell-local-dictionary: "croatian"
%%% TeX-master: "../hio.tex"
%%% End:
