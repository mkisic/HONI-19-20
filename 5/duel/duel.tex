%%%%%%%%%%%%%%%%%%%%%%%%%%%%%%%%%%%%%%%%%%%%%%%%%%%%%%%%%%%%%%%%%%%%%%
% Problem statement
\begin{statement}[
  problempoints=30,
  timelimit=1 sekunda,
  memorylimit=512 MiB,
]{Duel}

%\setlength\intextsep{-0.1cm}
%\begin{wrapfigure}[3]{r}{0.26\textwidth}
%\centering
%\includegraphics[width=0.20\textwidth]{img/steak.png}
%\end{wrapfigure}

Unatoč tome što su jako dobri prijatelji, Fabijan i Patrik su oduvijek rivali u
stvarnom svijetu programiranja i virtualnom svijetu popularne igre
\textit{FIFA 20}. Od Božića su odigrali $824$ utakmice, a
trenutni je rezultat $412:412$. Zaključili su da su u virtualnom svijetu
nogometa podjednako dobri. Sada je na redu borba za titulu najboljeg programera.
Dogovorili su se da će duelom odlučiti, jednom za svagda, tko je bolji
programer, a čija mačka crnu vunu prede.

Za duel su pripremili $N$ zadataka. Od $N$ zadataka Patrik je točno riješio
njih $P$, a Fabijan $F$. Sada svakog od njih zanima koliko je zadatka on
točno riješio, a koje njegov rival nije.

%%%%%%%%%%%%%%%%%%%%%%%%%%%%%%%%%%%%%%%%%%%%%%%%%%%%%%%%%%%%%%%%%%%%%%
% Input
\subsection*{Ulazni podaci}
U prvom je retku prirodan broj $N$ $(1 \le N \le 10^{18})$ iz teksta zadatka.

U drugom je retku cijeli broj $P$ $(0 \le P \le min(1000, N))$ iz teksta
zadatka.  Slijedi $P$ različitih brojeva $P_i$ $(1 \le P_i \le N)$, svaki u
svom retku, indeksi zadataka koje je Patrik uspješno riješio.

U sljedećem je retku cijeli broj $F$ $(0 \le F \le min(1000, N))$ iz teksta
zadatka.  Slijedi $F$ različitih brojeva $F_i$ $(1 \le F_i \le N)$, svaki u
svom retku, indeksi zadataka koje je Fabijan uspješno riješio.

%%%%%%%%%%%%%%%%%%%%%%%%%%%%%%%%%%%%%%%%%%%%%%%%%%%%%%%%%%%%%%%%%%%%%%
% Output
\subsection*{Izlazni podaci}
U prvi redak ispišite koliko je zadataka Patrik riješio, a Fabijan nije.\\
U drugi redak ispišite koliko je zadataka Fabijan riješio, a Patrik nije.


%%%%%%%%%%%%%%%%%%%%%%%%%%%%%%%%%%%%%%%%%%%%%%%%%%%%%%%%%%%%%%%%%%%%%%
% Scoring
\subsection*{Bodovanje}
U testnim primjerima vrijednima $6$ boda vrijedit će $N = 3$ i $P = F = 2$. \\
U testnim primjerima vrijednima dodatnih $12$ bodova vrijedit će $N \le 1000$.

%%%%%%%%%%%%%%%%%%%%%%%%%%%%%%%%%%%%%%%%%%%%%%%%%%%%%%%%%%%%%%%%%%%%%%
% Examples
\subsection*{Probni primjeri}
\begin{tabularx}{\textwidth}{X'X'X}
\sampleinputs{test/duel.dummy.in.1}{test/duel.dummy.out.1} &
\sampleinputs{test/duel.dummy.in.2}{test/duel.dummy.out.2} &
\sampleinputs{test/duel.dummy.in.3}{test/duel.dummy.out.3}
\end{tabularx}

  %\textbf{Pojašnjenje prvog probnog primjera:}

%21. siječnja je 21. dan u godini. Radi se o neparnom danu pa Ivan jede mrkvu.\\
%19. veljače je 49. dan u godini. Radi se o neparnom danu pa Ivan jede mrkvu.\\
%16. lipnja je 166. dan u godini. Radi se o parnom danu pa Ivan jede brokulu.\\

%%%%%%%%%%%%%%%%%%%%%%%%%%%%%%%%%%%%%%%%%%%%%%%%%%%%%%%%%%%%%%%%%%%%%%
% We're done
\end{statement}

%%% Local Variables:
%%% mode: latex
%%% mode: flyspell
%%% ispell-local-dictionary: "croatian"
  %%% TeX-master: "../hio.tex"
%%% End:
