%%%%%%%%%%%%%%%%%%%%%%%%%%%%%%%%%%%%%%%%%%%%%%%%%%%%%%%%%%%%%%%%%%%%%%
% Problem statement
\begin{statement}[
problempoints=70,
timelimit=1 second,
memorylimit=512 MiB,
]{Političari}

All politicians of an unknown, completely invented and totally unrealistic
country are spending their time accusing each other on national television
instead of doing their jobs. It all started one Sunday afternoon when
politician \textbf{number 1} was a guest in the first episode of a (now very
popular) talk show. During the show, he accused the politician \textbf{number
2} for the poor state of the country. Naturally, in the second episode of the
show the guest was politician number 2. The talk show host told his guest that
politician number 1 accused him and politician number 2 then blamed some other
politician. The newly blamed politician was the guest in the next show where the
host told him that\dots

Even today, after almost $20$ years, a new politician is a guest in each episode of the show where
he is being told by whom he was accused for the poor state in the country. That politician
then blames another politician and the vicious cycle continues. To make things
more interesting, we have exclusively found out that each politician has a fixed
strategy on how to behave during the show. More precisely, each politician knows
who to blame based on the person who blamed him in previous show. We will provide
you with this information and hope you will be able to write a program that
calculates what politician will be the guest of the $K$-th show.

%%%%%%%%%%%%%%%%%%%%%%%%%%%%%%%%%%%%%%%%%%%%%%%%%%%%%%%%%%%%%%%%%%%%%%
% Input
\subsection*{Input}
The first line contains integers $N$ $(2 \le N \le 500)$ and $K$
$(1 \le K \le 10^{18})$ from the task description.

The $i$-th of the next $N$ lines contains $N$ integers where $j$-th
integer tells us who will be blamed by the $i$-th politician if he was
blamed by politician number $j$ in the last show.

You can assume that no politician will ever blame himself. Therefore,
none of the numbers in $i$-th line of matrix will be equal to $i$. Similarly, note
that the $i$-th number in the $i$-th matrix row will always be equal to $0$ and
can be disregarded.

%%%%%%%%%%%%%%%%%%%%%%%%%%%%%%%%%%%%%%%%%%%%%%%%%%%%%%%%%%%%%%%%%%%%%%
% Output
\subsection*{Output}
In a single line you should output the number of a politician that will be
the guest of the $K$-th episode of the talk show.

%%%%%%%%%%%%%%%%%%%%%%%%%%%%%%%%%%%%%%%%%%%%%%%%%%%%%%%%%%%%%%%%%%%%%%
% Scoring
\subsection*{Scoring}
In the test cases worth a total of $35$ points, it will hold $1 \le K \le 10^5$.

%%%%%%%%%%%%%%%%%%%%%%%%%%%%%%%%%%%%%%%%%%%%%%%%%%%%%%%%%%%%%%%%%%%%%%
% Examples
\subsection*{Examples}
\begin{tabularx}{\textwidth}{X'X'X}
\sampleinputs{test/politicari.dummy.in.1}{test/politicari.dummy.out.1} &
\sampleinputs{test/politicari.dummy.in.2}{test/politicari.dummy.out.2} &
\sampleinputs{test/politicari.dummy.in.3}{test/politicari.dummy.out.3}
\end{tabularx}

%%%%%%%%%%%%%%%%%%%%%%%%%%%%%%%%%%%%%%%%%%%%%%%%%%%%%%%%%%%%%%%%%%%%%%
% We're done
\end{statement}

%%% Local Variables:
%%% mode: latex
%%% mode: flyspell
%%% ispell-local-dictionary: "croatian"
%%% TeX-master: "../hio.tex"
%%% End:
