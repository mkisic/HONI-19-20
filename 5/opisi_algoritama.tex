\documentclass[a4paper]{article}
\usepackage{zadaci}
\usepackage{wrapfig}
\usepackage{url}
\usepackage{tikz}
\usepackage{amsmath}
\usepackage[normalem]{ulem}
\usetikzlibrary{angles,quotes}
\contestname{Hrvatsko otvoreno natjecanje u informatici\\5.\ kolo, 8. veljače 2020.}
\markright{\textbf{\textsf{Opisi algoritama}}}

\usepackage{hyperref}
\hypersetup{
colorlinks=true,
linkcolor=blue,
filecolor=magenta,
urlcolor=cyan,
}

\begin{document}

\section*{Opisi algoritama}
Zadatke, testne primjere i rješenja pripremili: Nikola Dmitrović, Karlo Franić,
Gabrijel Jambrošić, Marin Kišić, Daniel Paleka, Ivan Paljak i Paula Vidas.
Primjeri implementiranih rješenja su dani u priloženim izvornim kodovima.

\subsection*{Zadatak: Steak}
\textsf{Pripremio: Nikola Dmitrović}\\
\textsf{Potrebno znanje: naredba učitavanja brojeva i riječi, naredba
ispisivanja, naredba odlučivanja, naredba ponavljanjaj}

Na prvo čitanje se čini da ćemo za redni broj dana u godini trebati uzeti u
obzir i mjesec, tj. da ćemo prvo morati odrediti redni broj zadanog mjeseca,
taj redni broj pomnožiti s $30$ i na to dodati datum. Npr. dan \texttt{16 LIPANJ}
je $30 \cdot 6 + 16 = 196$.
dan u godini (lipanj je šesti mjesec).

Međutim, kako je dogovoreno da će svaki mjesec imati $30$ dana, možemo
zaključiti da se parnost/neparnost dana $D$ iz teksta zadatka neće mijenjati u
ovisnosti o mjesecu u kojem se nalazi. Primjerice, ako promatramo slučaj kada
je $D=16$, redni broj toga dana u siječnju će biti $16$, u veljači $46$, u
ožujku $76$, itd., sve parni brojevi kao što je i sam broj $16$.

U testnim primjerima vrijednima 10 bodova nije bilo nužno koristiti naredbu ponavljanja.

\textit{Programski kod (pisan u \texttt{Python 3}):}

\vspace{-2ex}
\begin{verbatim}
N = int(input())
for i in range(N):
    D = int(input())
    M = input()
    if D % 2 == 0:
        print('BROKULA')
    else:
        print('MRKVA')
\end{verbatim}

\subsection*{Zadatak: Duel}
\textsf{Pripremili: Karlo Franić i Ivan Paljak}\\
\textsf{Potrebno znanje: jednodimenzionalna polja}

Za $6$ bodova bilo je dovoljno provjeriti jesu li Fabijan i Patrik riješili oba
ista zadatka ili je svatko riješio jedan koji drugi nije uspio riješiti.

Za dodatnih $12$ bodova bilo je potrebno stvoriti dva polja (jedno za Patrika, a
drugo za Fabijana) u kojima ćemo s $1$ označiti riješen zadatak, a s $0$ onaj koji
nije riješen. Zatim smo mogli naredbom ponavljanja proći kroz oba polja te u
svakom prebrojati koliko ima jedinica, a da se na istoj poziciji u drugom
polju nalazi nula.

Za sve bodove u polje je bilno potrebno spremati indekse zadataka te
naredbom ponavljanja za svaki zadatak koji je Patrik riješio provjeriti nalazi
li se isti zadatak u popisu Fabijanovih zadataka. Isti proces tada ponavljamo
za Fabijanove zadatke.

Također za zadatak postoji alternativno rješenje koje prebroji koliko
su istih zadataka riješili. Odgovor na pitanje zašto nam je ta informacija
korisna ostavljamo čitateljici za vježbu.

\clearpage

\subsection*{Zadatak: Emacs}
\textsf{Pripremila: Paula Vidas}\\
\textsf{Potrebno znanje: naredba ponavljanja, rad sa stringovima}

Broj pravokutnika na slici jednak je broju gornjih lijevih kuteva pravokutnika.
Polje $(i, j)$ je gornji lijevi kut nekog pravokutnika ako u tom polju piše
\texttt{'*'}, a u poljima $(i - 1, j)$ i $(i, j - 1)$ piše \texttt{'.'}.
Naravno, moramo dodatno pripaziti na polja u prvom retku i prvom stupcu.

Alternativno, možemo BFS algoritmom pronaći broj povezanih područja znakova
\texttt{'*'}.

\subsection*{Zadatak: Političari}
\textsf{Pripremio: Ivan Paljak}\\
\textsf{Potrebno znanje: iskorištavanje cikličnosti}

Na ovom zadatku bilo je moguće osvojiti $50\%$ ($35$) bodova naivnom implementacijom
onog što piše u tekstu zadatka. Odnosno, bilo je potrebno ispravno
implementirati pravila kojima se političari međusobno optužuju te tako saznati
koji će se političar pojaviti u $K$-toj emisiji. Vremenska složenost ovog
pristupa je $\mathcal{O}(K)$, a implementaciju možete vidjeti u izvornom kodu
\texttt{politicari\_brute.cpp}.

Za osvajanje svih bodova bilo je potrebno primijetiti da je proces cikličan.
Budući sljedeći gost ovisi o isključivo tome tko je bio prethodni gost te tko
je tog gosta optužio, zaključujemo da postoji $\mathcal{O}(N^2)$ različitih emisija
(stanja). Dakako, dvije emisije smatramo jednakima ako u njima gostuje isti
političar koji je optužen od strane iste osobe. U protivnom, emisije smatramo
različitima.

Budući da je broj stanja znatno manji od rednog broja emisije koja nas zanima u
tekstu zadatka, možemo simulirati proces iz zadatka sve dok se ponovno ne
dogodi emisija koju smo već vidjeli. U tom trenutku, ako smo do sad zapamtili
neke relevantne informacije, možemo metodama matematike i računanja odrediti
tko će gostovati u $K$-toj emisiji.

Pretpostavimo da smo simulirajući proces primijetili da je $i$-ta emisija jednaka
nekoj prethodnoj $j$-toj emisiji. U tom slučaju smo ušli u ciklus veličine $(i-j)$ te
zaključujemo da je gost $K$-te emisije jednak gostu koji se pojavio u $(j + ((K -
j) \% (i - j)))$-toj emisiji, gdje znakom $\%$ označavamo operator modulo.

Vremenska složenost ovakvog algoritma je $\mathcal{O}(N^2)$.

\subsection*{Zadatak: Matching}
\textsf{Pripremili: Paula Vidas i Daniel Paleka}\\
\textsf{Potrebno znanje: tournament stablo}

Ako povežemo točke koje imaju jednaku jednu koordinatu, točke će biti
podijeljene u \textit{cikluse} i \textit{puteve}. Za puteve je određeno koje
točke moramo spojiti. Ako postoji put sa neparno mnogo točaka odgovor je odmah
\texttt{"NE"}. Za cikluse moramo odrediti hoćemo li povući sve vodoravne ili sve okomite
dužine (u nastavku teksta to zovemo \textit{orijentacijom}).

Ciklusa ima najviše $\frac{N}{4}$, pa za podzadatak $(N \le 40)$ možemo
isprobati svih $2^{\text{broj ciklusa}}$ mogućnosti za orijentacije ciklusa, i
provjeriti postoji li neka mogućnost u kojoj se dužine ne sijeku.

Glavna ideja je sljedeća: na početku putevi određuju neke dužine koje moramo
nacrtati, pa onda te dužine određuju orijentacije onih ciklusa koje sijeku i
crtamo te dužine, itd. Ako u nekom trenutku nacrtamo dužinu koja siječe neku
već nacrtanu ili dobijemo da neki ciklus ne može biti niti okomit niti
vodoravan, onda je odgovor \texttt{"NE"}.

Za podzadatak $N \leq 2000$ to možemo napraviti u složenosti $O(N^2)$:

Nađemo puteve i cikluse, te dužine iz puteva ubacimo u \textit{queue} dužina koje treba
nacrtati. Sve dok queue nije prazan, uzimamo dužinu iz queuea. Provjerimo
siječe li se s nekim već nacrtanim dužinama (moguće je i da smo za neki ciklus
nacrtali obje orijentacije, pa će se sijeći u točki, i tada je odgovor \texttt{"NE"}).
Zatim prođemo po svih ciklusima kojima još nismo odredili orijentaciju, te ako
dužina siječe ciklus, moramo ga orijentirati suprotno od dužine i ubaciti te
nove dužine u queue. Na kraju se eventualno može dogoditi da za neke cikluse
nismo odredili orijentaciju, pa njih sve orijentiramo jednako.

Za $N \le 10^5$, moramo pametnije napraviti provjeru siječe li se dužina s nekim
drugim dužinama. Pomoću tournament stabla napravit ćemo strukturu koja podržava:

\begin{itemize}
  \item dodaj dužinu $(x_1, y)$ -- $(x_2, y)$
  \item makni dužinu $(x_1, y)$ -- $(x_2, y)$ (koja je nekada prije ubačena)
  \item odredi za neke $x$ i $y_1$ najmanji $y_2 \ge y_1$ takav da postoji
    dužina s $y$ koordinatom jednakom $y_2$ koja sadrži točku $(x, y_2)$ (ili
    odredi da takav $y_2$ ne postoji)
\end{itemize}

U čvoru tournamenta pamtit ćemo skup svih y koordinata dužina kojima su x
koordinate u intervalu za koji je odgovoran taj čvor. Prva dva upita su
ubacivanje (odnosno izbacivanje) $y$ u određene čvorove. Za odgovor na treći
upit, nađemo takav $y_2$ za svaki čvor odgovoran za $x$ (pomoću
\texttt{lower\_bound}-a u \texttt{std::set}-u), i uzmemo minimalni. Složenost
sva tri upita je $\mathcal{O}(\log^2 N)$. Za implementaciju pogledajte
službeni kod.

Dužina koja siječe dužinu $(x, y_1) - (x, y_2)$ postoji ako je odgovor na treći
upit za $x$ i $y_1$ manji ili jednak od $y_2$.

Imat ćemo dvije strukture -- za okomite i za vodoravne dužine -- u kojima ćemo
spremiti dužine još neorijentiranih ciklusa, te dvije strukture u kojima
spremamo nacrtane dužine. Razradu detalja ostavljamo čitatelju na vježbu.

\subsection*{Zadatak: Putovanje}
\textsf{Pripremili: Gabrijel Jambrošić i Marin Kišić}\\
\textsf{Potrebno znanje: lca, manje na veće}

Primjetimo da je pravi problem u zadatku saznati koliko puta ćemo proći
svakom cestom. Kada to znamo, lako možemo odrediti želimo li kupiti višekratnu
ili jednokratnu kartu za svaku cestu.

Za prvi podzadatak bilo je dovoljno samo nekim algoritmom poput dfs-a ili bfs-a
pronaći put između $X$ i $X+1$ te brojač svih cesta na putu povećati za $1$.

U drugom podzadatku nemamo stablo, nego lanac. Zapišimo taj lanac kao niz. Sada
za svaki par $X$ i $X+1$ možemo u nekom nizu \texttt{dp} na poziciji
\texttt{min(poz[X], poz[X+1])} povećati vrijednost za $1$, a na poziciji
\texttt{max(poz[X], poz[X+1])} smanjiti vrijednost za $1$. Nakon što smo to
napravili, za sve $X$ od $1$ do $(N-1)$, imat ćemo varijablu \texttt{cnt},
prolaziti nizom \texttt{dp} te \texttt{cnt} mjenjati za vrijednost
\texttt{dp[i]} (\texttt{cnt += dp[i]}). Primjetimo da zapravo sada u svakom
trenu u varijabli \texttt{cnt} piše koliko puta smo tu cestu prešli.

Za sve bodove ćemo ovaj algoritam prilagoditi da radi na stablu.
Ukorijenimo stablo u proizvoljnom čvoru. Neka je $L$ prvi zajednički predak
(\textit{lowest common ancestor (LCA)}) od $X$ i $(X+1)$. Povećajmo sada
\texttt{dp[X]} za $1$, \texttt{dp[X+1]} za $1$,
te smanjimo \texttt{dp[L]} za $2$. Sada kada bismo za neku cestu uzeli sumu svih
\texttt{dp}
vrijednosti u podstablu te ceste zapravo bismo dobili koliko puta smo prošli tu
cestu.

Ovo rješenje je implementirano u \texttt{putovanje\_lca.cpp}

Vremenska složenost je $\mathcal{O}(N \log N)$.

Postoji i alternativno rješenje iste vremenske složenosti koje koristi \textit{
  manje na veće} trik, a implementacija tog rješenja se nalazi u
\texttt{putovanje\_stl.cpp}, a za idejske detalje pogledajte rješenje jako
sličnog zadatka opisano na ovom
\href{https://codeforces.com/blog/entry/72017#comment-563190}{linku}.

\subsection*{Zadatak: Zapina}
\textsf{Pripremili: Marin Kišić i Paula Vidas}\\
\textsf{Potrebno znanje: dinamičko programiranje, prebrojavanje}

Prvi podzadatak možemo riješiti tako da za svaki mogući raspored provjerimo
postoji li sretan informatičar. Mogućih rasporeda ima $N^N$.

Drugi podzadatak možemo riješiti koristeći formulu uključivanja-isključivanja.
Ako s $a(S)$, gdje je $S \subset \{1, 2, \ldots, N\}$, označimo broj rasporeda
u kojima su svi informatičari iz skupa $S$ sretni, tada je broj dobrih
rasporeda jednak

$$\sum_{S \subset \{1, 2, \ldots, N\}} (-1)^{|S| + 1} a(S)$$.

Neka je $S = {s_1, s_2, \ldots, s_k}$. Ako je $s_1 + s_2 + \cdots + s_k > N$
onda je očito $a(S) = 0$. Inače, vrijedi 

$$a(S) = \binom{n}{s_1} \binom{n - s_1}{s_2} \cdots \binom{n - (s_1 + s_2 +
\cdots s_{k-1}}{s_k} (n - k)^{n - (s_1 + s_2 + \cdots + s_k)}$$.

Povrhe možemo preprocessati koristeći relaciju $\binom{n}{k} = \binom{n - 1}{k
- 1} + \binom{n - 1}{k}$, u složenosti $\mathcal{O}(N^2)$.

Ukupna vremenska složenost je $\mathcal{O}(N 2^N)$.

Treći podzadatak rješavamo dinamičkim programiranjem. Neka je $dp(n, k)$ broj
raspodjela $k$ zadataka na $n$ informatičara u kojima je bar jedan informatičar
sretan. Imamo dvije mogućnosti -- hoćemo li $n$-tom informatičaru dati točno $n$
zadataka (pa da bude sretan), ili nećemo. U prvom slučaju je broj načina jednak
(ako je $k \leq n$)

$$\binom{k}{n} (n - 1)^{(k - n)},$$

a u drugom

$$\sum_{0 \leq i \leq k, i \neq n} \binom{k}{i} dp(n - 1, k - i)$$.

Ukupna vremenska složenost je $\mathcal{O}(N^3)$.

\end{document}
%%% Local Variables:
%%% mode: latex
%%% mode: flyspell
%%% ispell-local-dictionary: "croatian"
%%% End:
